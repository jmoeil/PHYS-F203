\documentclass[../notesdecours.tex]{subfiles}

\begin{document}
    
\chapter{Notions mathématiques}

\paragraph{} Dans ce chapitre, nous allons rapidement passer en revue plusieurs notions de maths, 
plus particulièrement d'analyse, qui apparaissent dans le cours de Mécanique Quantique, 
afin de vous aider dans la compréhension de certains passage de calculs. \\
Nous allons commencer par de brefs rappels de CDI2 sur les séries et transformées de Fourier, 
pour ensuite embrayer sur la notion de distribution, qui est certainement neuve dans votre parcours,
mais cela permettra plus tard de justifier l'utilisation de distribution pour décrire les fonctions d'onde. \\

\section{Séries de Fourier}
Considérons $f$ une fonction T-périodique ; \\

\begin{definition}[coefficients de Fourier]
    On définit ses coefficients de Fourier (exponentiels) par la formule : 
\begin{equation}
c_n = \frac{1}{T} \int^{T/2}_{-T/2} dx f(x)e^{-2\pi i(\frac{n}{T})x} \\
\end{equation}
\end{definition}

\begin{definition}[Série de Fourier]
    On appelle la série de Fourier associée à $f$, la série de fonctions qui est telle que 
$\forall n \in \mathbb{N}$, la somme partielle de cette série $S_n(f)$ est donnée par :
\begin{equation}
    \forall n\in \mathbb{N} ; \quad \forall x \in \mathbb{R} ;  \quad S_n(f)(x) = \sum_{k=-n}^{n} c_k(f) e^{2\pi i \left( \frac{k}{T} \right) x} \\
\end{equation}
\end{definition}

\begin{theorem}[Théromème de Dirichlet]
    Le théorème de Dirichlet (global) nous assure que la série de Fourier de $f$ converge uniformément 
vers la fonction $f$ sur $\mathbb{R}$, ainsi nous pouvons écrire :  
\begin{equation}
    \forall x \in \mathbb{R} ; \quad f(x) = \sum_{k = -\infty}^{+\infty}c_ne^{2\pi i (\frac{k}{T})x} \\
\end{equation}
Notons que nous pouvons appliquer ce théorème car en physique, nous ne considérons en général que des fonctions très lisses (de classe \textit{C}$^{\infty}$).
\end{theorem}



\section{Transformées de Fourier}

Considérons une fonction $f$ (dans le cours de CDI2, nous faisons l'hypothèse que la fonction 
appartient à la classe de Schwartz, mais en réalité, la transformée de Fourier et ses propriétés se généralisent 
à des fonctions bien moins lisses). \\

\begin{definition}[Transformée de Fourier]
    La fonction  $\mathcal{F}(f)$, que l'on note aussi ici $\hat{f}(k)$, est la transformée 
    de Fourier de la fonction $f$, et est définie par : 
\begin{equation}
\hat{f}(k) = \mathcal{F}(f) = \frac{1}{\sqrt{2 \pi}} \int_{-\infty}^{+\infty} dx f(x) e^{-ikx}
\end{equation}
\end{definition}

\begin{theorem}[Inversion de la transformée de Fourier] 
    $\forall x \in \mathbb{R}$, on a $\mathcal{F}(\mathcal{G}(f))(x) = f(x)$ ou $\mathcal{G}(\mathcal{F}(f))(x) = f(x)$, 
    où $\mathcal{G}(f) = \mathcal{F}^{-1}(f)$ est l'inverse de la transformée de fourier de $f$. \\
    Ceci s'exprime encore comme : 
    \begin{equation}
        f(x) = \mathcal{G}(\hat{f})(x) = \frac{1}{\sqrt{2 \pi}} \int_{-\infty}^{+\infty} dk\hat{f}(k) e^{ikx}
    \end{equation}
Remarquons que $\mathcal{G}(f)(k)$ n'est rien d'autre que $\mathcal{F}(f)(-k)$
\end{theorem}

\begin{remark}
Si $f$ est à support borné et $\{-\frac{T}{2},\frac{T}{2}\}$ contraint le support, alors $C_n = \frac{\sqrt{2\pi}}{T}\hat{f}(\frac{2\pi n}{T})$.
\end{remark}
\begin{align}
\rightarrow f(x) &= \sum_{n = -\infty}^{+\infty} \frac{\sqrt{2\pi}}{T}\hat{f}(\frac{2\pi n}{T})e^{i\frac{2\pi n}{T}x}\\
&= \sum_{n = -\infty}^{+\infty} \hat{f}(k_n)\frac{e^{ik_nx}}{\sqrt{2\pi}} \Delta k		&k_n = \frac{2\pi n}{T} \textbf{ et } \Delta k = \frac{2\pi}{T}\\
&\approx \int_{-\infty}^{+\infty} dk \hat{f}(k) \frac{e^{ikx}}{\sqrt{2\pi}}
\end{align}
\begin{remark}
Ces fonctions suivent certaines propriétés intéressantes. Soit $h(x)$ et $\hat{h}(x)$ deux fonctions reliées par une transformations de Fourier. Dès lors,
\begin{itemize}
\item Si $h(x)$ est linéaire, alors $\hat{h}(x)$ l'est également : $h(x) = af(x) + bg(x)$, alors $\hat{h}(k) = a\hat{f}(k) + b\hat{g}(k)$.
\item Si $h(x) = f(x-x_0)$, alors $\hat{h}(k) = e^{-ikx_0}\hat{f}(k)$. Il s'agit d'une translation. Inversement, la propriété de modulation s'écrit $h(x) = f(x)e^{ik_0x}$, alors $\hat{h}(k) = \hat{f}(k-k_0)$.
\item Si $h(x) = f(ax)$, le changement d'échelle implique que $\hat{h}(k) = \frac{1}{a}\hat{f}(\frac{k}{a})$.
\item La relation de conjuguaison sous une transformation de Fourier est que $h(x) = \bar{f}(x)$ implique $\hat{h}(k) = \bar{\hat{f(-k)}}$. Notons que si $f(x)$ est réel, alors $\hat{f}(k) = -\hat{f}(k)$.
\item $\hat{f}(0) = \int_{-\infty}^{\infty} dxf(x)$.
\item La dérivée de $\hat{f}(k)$ est $ik\hat{f}(k)$. Cela se généralise à $\hat{f^(n)} = (ik)^n\hat{f}(k)$. En particulier, si $f(x)x^n$ est intégrable, alors $\hat{f}(k)$ est n-fois dérivable. Inversement, si $f(x)$ est n-fois intégrable, alors $\hat{f}(k)k^n$ est intégrable.
\item La propriété de convolution établit que si $h(x) = (f\circ h)(x) = \int dy f(y)g(x-y)$, alors $\hat{h}(k) = \hat{f}(k)\ti\hat{g}(k)$.
\end{itemize}

\end{remark}
\begin{theorem}[Plancherel]
Soit $f(x)$ une fonction, et $\hat{f}(k)$ sa transformée de Fourier. Nous avons alors l'équivalence des intégrales:
\begin{equation}
\int dx f(x)\bar{g}(x) = \int dk \hat{f}(k)\bar{\hat{g}}(k)
\end{equation}
\end{theorem}
\begin{theorem}[Égalité de Parceval]
Soit $f(x)$ une fonction, et $\hat{f}(k)$ sa transformée de Fourier. Alors,
\begin{equation}
\int dx \norm{f(x)}^2 = \int dk \norm{\hat{f}(k)}^2
\end{equation}
\end{theorem}

\section{Distribution}
\subsection{Espace de fonctions test}
Soient $D$, l'ensemble des fonctions $C^\infty$ à support compact (distrubution D'), et S - l'ensemble des fonctions $C^\infty$ à décroissance rapide (distrubtion tempérée S').  Imposons une notion de continuité/topologie sur les fonctions test:
\begin{equation}
\varphi_k = \varphi \text{ si et seulement si } (\partial_x^{(\alpha)} \varphi_x) = (\partial_x^{(\alpha)}\varphi)
\end{equation}
uniformément pour tout $\alpha$.\\

Soit T des formes linéaires continues sur l'espace des fonctions tests.
\begin{Property}
Soit $T: D \rightarrow \mathbb{R}$ : $\varphi \rightarrow T\cdot\varphi$. Si $\varphi_k = \varphi$, alors $T\cdot\varphi_k \rightarrow T\cdot\varphi$ généralise la notion de fonction.
\end{Property}

\subsection{Opérations sur les distributions}
\begin{Property}[Dérivée d'une distrubution]
$T'\cdot\varphi = T\cdot(-\varphi')$
\end{Property}
\begin{Property}[Multiplication d'une distribution par une fonction test]
$\Phi T\cdot\varphi = T\cdot\varphi\Phi$
\textbf{Nous ne pouvons pas multiplier des distributions entre-elles.}
\end{Property}
\begin{theorem}[Théorème de structure]
Localement, une distrubution est égale à la dérivée $\alpha^{eme}$ d'une fonction continue. Elle est dite tempérée lorsqu'elle est égale à la dérivée $\alpha^{eme}$ d'une fonction continue à croissance lente\footnote{ne croissant pas plus vite qu'un polynome.}.
\end{theorem}
\subsection{Distributions tempérées}
A partir de maintenant, nous noterons F une transformée de Fourier, et $\mathbb{S}$ une invariance sous F.
\begin{definition} Soit $T \in \mathbb{S}$. Alors, FT existe et est défini par $FT\cdot\Phi = T\cdot F\Phi$.
\end{definition}

Si f est une fonction, alors:
\begin{align}
FT_f\cdot\Phi &= T_f\cdot F\Phi		&\text{Où }\int dx (\int dx \frac{e^{-ikx}}{\sqrt{2\pi}}f(x))\Phi(k) \text{ et } \int dx f(x) (\int dk \frac{e^{-ikx}}{\sqrt{2\pi}}\Phi(k))
\end{align}

\subsection{Delta de Dirac}
\begin{align}
\delta(x) &= \begin{cases}
+\infty \mbox{ en x = 0}\\
0 \mbox{ en } x\neq 0
\end{cases}		&\int^{+\infty}_{-\infty} dx \delta (x) = 1\\
\delta(x) &= \lim_{x\to 0} f_{\alpha}(x)	&\int^{+\infty}_{-\infty} dx f_\alpha (x) = 1
\end{align}
Où $f_\alpha (x)$ est strictement positif.
\begin{align}
\int_{-\infty}^{+\infty} dx f(x)\delta(x) &= f(0)\\
\int_{-\infty}^{+\infty} dx \delta(\Gamma - x)\delta(x- \zeta) &= \delta (\Gamma - \zeta)\\
\delta '(x) : \int^{+\infty}_{-\infty} \delta '(x)f(x) &= [\delta(x)f(x)]_{-\infty}^{+\infty} - \int_{-\infty}^{+\infty} dx \delta(x)f'(x)\\
&= -f'(0)\\
\int_{-\infty}^{+\infty} dx' \delta(x') &= \theta(x)		&\int dx f(x)\delta(x-a) = f(a)\\
\delta(\alpha x) &= \frac{1}{\norm{\alpha}}\delta(x)	&\delta(g(x)) = \frac{1}{\norm{g'(x_0)}}\delta(x-x_0)
\delta(-x) &= \delta(x)\\
\end{align}

\subsection{Transformée de Fourier d'une fonction périodique}
Si $x(t)$ est une fonction de période T tel que $x(t+T) = x(t)$. Alors $x(t)$ peut-être représenté comme une série de Fourier.
\begin{equation}
x(t) = \sum_{k = -\infty}^{+\infty} c_ke^{2\pi ik\frac{i}{T}}
\label{Z}
\end{equation}
Prenons la transformée de Fourier de \eqref{Z}.
\begin{align}
\hat{x} (\omega ) = \int dt \frac{e^{-i\omega t}}{\sqrt{2\pi}} x(t) &= \int_{k = -\infty}^{+\infty} c_k \int dt \frac{e^{-i\omega t}}{\sqrt{2\pi}}e^{2\pi i k \frac{t}{T}}\\
&= \sum_{k = -\infty}^{+\infty} \frac{c_k}{2\pi} \delta (\omega - \frac{2\pi k}{T})
\end{align}
Nous appelons $\hat{x}(\omega)$ est la somme des deltas espacés de $\frac{2\pi}{T}$.

\end{document}