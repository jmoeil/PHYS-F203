\documentclass[../notesdecours]{subfiles}

\begin{document}
\part{L'équation de Schrödinger}
Commençons par rappeler les relations de \textbf{Plank-Einstein}:
\begin{align}
E = h \nu &= \h \omega\\
\bm{p} &= \h \bm{k}
\end{align}
Où $\lambda = \frac{2\pi}{\norm{\bm{k}}} = \frac{h}{\norm{\bm{p}}}$ : il s'agit de la relation de L. de Broglie, reflettant la dualité onde-corpusculaire de la matière.\\

Rappelons également l'équation d'onde :
\begin{equation}
\label{D'Alembert}
\left( \frac{1}{c^2} \partial_t^2 - \partial_x^2 - \partial_y^2 - \partial_z^2 \right)A(t,\bm{x}) = 0
\end{equation}
En particulier, une onde plane\footnote{Une onde est dite plane si et seulement si elle ne s'exprime que dans une seule direction.} s'exprime par le champ scalaire 
\begin{equation}
\label{Onde plane Sch}
A(t,\bm{x}) = A_0 e^{-i \left( \omega t - \bm{k} \cdot \bm{x} \right) }
\end{equation}
Appliquons l'équation d'une onde plane \ref{Onde plane} à l'équation de D'Alembert \ref{D'Alembert}. En particulier, notons que
\begin{center}
\begin{tabular}{c c}
$\frac{1}{c^2} \partial_t A(t,\bm{x}) = \frac{-iA_0 \omega}{c^2} e^{-i \left( \omega t - \bm{k} \cdot \bm{x} \right)}$ & $\frac{1}{c^2} \partial_t^2 A(t,\bm{x}) = - A_0 \left( \frac{\omega}{c}\right)^2 e^{-i \left( \omega t - \bm{k} \cdot \bm{x} \right)}$\\
\end{tabular}
\end{center}
Rappelons que nous considérons une onde plane : supposons que celle-ci se déplace dans la direction des x. Alors,
\begin{center}
\begin{tabular}{c c}
$\partial_x A(t,x) = iA_0k_x e^{-i \left( \omega t - xk_x \right)}$ & $\partial_x^2 A(t,x) = -A_0k_x^2 e^{-i \left( \omega t - xk_x \right)}$
\end{tabular}
\end{center}
Où $k_x$ est la composante en x du vecteur $\bm{k} = \left( k_x,k_y,k_z \right)$. En combinant ces résultats (et en gardant l'hypothèse d'une onde se dirigeant selon l'axe des x), nous obtenons que
\begin{align}
\left( \frac{1}{c^2} \partial_t^2 - \partial_x^2 - \partial_y^2 - \partial_z^2 \right)A(t,x) &= 0\\
- A_0 \left( \frac{\omega}{c}\right)^2 e^{-i \left( \omega t - xk_x \right)} +A_0k_x^2 e^{-i \left( \omega t - xk_x \right)} &= 0\\
\left( \frac{\omega}{c}\right)^2 - k_x^2 &= 0
\end{align}
A partir de là$\footnote{Demander à Prof. d'élaborer les détails.}$, nous pouvons montrer que 
\begin{align}
\omega &= \frac{i}{A} \partial_t \left( A \right)\\
k_x &= - \frac{i}{A} \partial_x \left( A \right)
\end{align}
Ce qui implique les relations
\begin{center}
\begin{tabular}{c c c}
$k_x^2 = - \frac{1}{A} \partial_x^2 A$ & et & $k^2 = - \frac{1}{A} \Delta A$
\end{tabular}
\end{center}

On pose également les quelques relations suivantes
\begin{center}
\begin{tabular}{c c c}
$E = \frac{i \h}{\Psi} \partial_t \Psi$ & $p_X = - \frac{i \h}{\Psi} \partial_x \Psi$ & $p^2 = - \frac{\h^2}{\Psi} \Delta \Psi$
\end{tabular}
\end{center}
Dans le cas d'une particule relativiste, l'énergie respecte la relation de dispersion. En particulier, cela implique
\begin{equation}
\label{Klein-Gordon}
- \h \partial_t \Phi + c^2 \h^2 \Delta \Phi - c^4m^2\Phi = 0
\end{equation}
Il s'agit de l'équation de Klein-Gordon. Nous pouvons obtenir une relation similaire dans le cas non-relativiste. Effectivement, nous avons alors que l'énergie est donnée par la relation
\begin{equation}
E = \frac{p^2}{2m} + V(\bm{r},t)
\end{equation}
Dès lors, nous obtenons l'équation de Schrödinger pour une particule:
\begin{equation}
\label{Schrodinger}
i \h \frac{\partial \Psi}{\partial t} = -\frac{\h^2}{2m} \Delta \Psi + V(\bm{r},t) \Psi \doteq H \Psi
\end{equation}
\newpage
\textbf{Note.} Nous avons alors que l'opérateur $H \doteq -\frac{\h^2}{2m} \Delta + V (\bm{r},t)$.
Remarquons que:
\begin{itemize}
\item L'équation de Schrödinger est linéaire\footnote{Si $\Psi_1$ et $\Psi_2$ sont des solutions de \ref{Schrodinger}, alors $\Psi_1 + \Psi_2$ est également une solution de \ref{Schrodinger}.} et satisfait les relations de Broglie.
\item L'opérateur H est hermitien ; cela garantit la conservation de la probabilité, que ses valeurs propres sont réelles et que ses vecteurs propres constituent une base de l'espace considéré. 
\end{itemize}

\section{Particule libre}

Dans une première approximation, supposons que le potentiel $V(\bm{r},t)$ soit nul. L'équation de Schrödinger ($\ref{Schrodinger}$) se réduit alors à $i\h \partial_t \Psi = - \frac{\h^2}{2m} \Delta \Psi$. De plus, l'énergie s'exprime alors par $E = \frac{p^2}{2m}$. Cette équation différentielle admet visiblement des solutions de la forme
\begin{equation}
\label{Solution directe}
\Psi (\bm{r},t) = Ce^{i(\bm{k} \cdot \bm{x} - \omega t)}
\end{equation}

Où C est une constante, et $\omega = \frac{\h \norm{\bm{k}}^2}{2m}$. Par la principe de superposition, toute combinaison linaire d'ondes planes vérifiant l'expression de $\omega$ sera également une solution de $\ref{Solution directe}$. La solution générale est alors donnée par 
\begin{equation}
\label{Solution générale}
\Psi (\bm{r},t) = \frac{1}{\left( 2 \pi \right)^{\frac{3}{2}}}\int d^3k g(\bm{k}) e^{-i \omega t + \bm{k} \cdot \bm{r}}
\end{equation}
Où g($\bm{k}$) est la transformation de Fourier de $\Psi (\bm{r},t = 0)$:
\begin{equation}
g(\bm{k},t=0) = \frac{1}{\sqrt{2\pi}} \int \Psi (\bm{r},0)e^{-i\bm{k}x} dx
\end{equation}

\section{Interprétation probabiliste}

Soit $\Psi (\bm{r},t)$ une solution de ($\ref{Schrodinger}$). Nous pouvons vérifier que $\Psi$ est normaliée : effectivement, $\int d\bm{r} \norm{\Psi(\bm{r},t)}^2 = 1$. Il est possible de montrer que si c'est le cas à un instant donné, ça l'est à tout moment.\\

Posons donc que $\rho (\bm{r},t) = \norm{\Psi (\bm{r},t)}^2 = \bar{\Psi} \Psi$ la densité de probabilité de trouver la particule en $\bm{r}$ à l'instant t.\\

\begin{Property}
La densité de probabilité telle que nous venons de la définir respecte l'équation de continuité
\begin{equation}
\label{Equation de Continuité}
\partial_t \rho (\bm{r},t) + \bm{\nabla} \cdot \bm{J} (\bm{r},t) = 0
\end{equation}
Où $\bm{J} (\bm{r},t) \dot{=} \frac{\h}{2mi} \Big\{\overline{\Psi}\bm{\nabla}\Psi - \Psi\bm{\nabla}\overline{\Psi}\Big\} = \frac{\h}{m} Im \left[\bar{\Psi} \left(\bm{\nabla} \Psi\right)\right]$ est un courant de probabilité.\\
\end{Property}

\begin{proof}
Puisque $\Psi$ est hermitique,
\begin{equation}
\hat{H}\Psi = \hat{H}\bar{\Psi}
\end{equation}
\begin{align}
i\h\partial_t\bar{\Psi} &= - \frac{\h^2}{2m}\Delta\bar{\Psi}+V\bar{\Psi}	&i\h\partial_t\Psi = -i\h\partial_t\bar{\Psi}
\end{align}
De plus, par définition,
\begin{equation}
\rho = \norm{\Psi}^2 = \Psi\bar{\Psi}
\end{equation}
En dérivant $\rho$ par rapport à au temps et en multipliant par $i\h$,
\begin{align}
i\h\partial_t\rho &= \left[-\frac{\h^2}{2m}\Delta \Psi\right]\bar{\Psi} - \left[-\frac{\h^2}{2m}\Delta\bar{\Psi}\right]\Psi\\
&= -\frac{\h^2}{2m} (\bar{\Psi}\Delta\Psi - \Psi\Delta\bar{\Psi})\\
\rightarrow 0 &= \partial_t\rho + \frac{\h}{2mi}\bm{\nabla}\cdot(\bar{\Psi}\bm{\nabla}\Psi - \Psi\bm{\nabla}\bar{\Psi})
\end{align}
Nous pouvons alors définir 
\begin{equation}
\bm{J} \dot{=} \frac{\h}{2mi} \Big\{\overline{\Psi}\bm{\nabla}\Psi - \Psi\bm{\nabla}\overline{\Psi}\Big\} = \frac{\h}{m} Im \left[\bar{\Psi} \left(\bm{\nabla} \Psi\right)\right]
\label{Courant de probabilité}
\end{equation}
le courant de probabilité. Nous pouvons alors écrire l'équation de probabilité, ce qui conclut la preuve.
\end{proof}

L'équation de continuité $\eqref{Equation de Continuité}$ explique que rien ne se perd, rien ne se crée et tout se conserve : en effet, si nous intégrons sur une région A, nous aurons
\begin{align}
\int_A \frac{\partial \rho}{\partial t} dV + \int_A \bm{\nabla}\cdot\bm{J} dv &= \frac{d}{dt} \int_A \rho dV + \int_{\partial A} \bm{J}\cdot\bm{ds}\\
\frac{d}{dt} P_A &= -F_A \label{Conservation des particules}
\end{align}
Où $P_A$ est la probabilité de trouver une particule dans la région A, et $F_A$ est le flux de cette particule à travers le bord de la région A. L'équation \eqref{Conservation des particules} décrit la conservation des particules à travers la surface A.\\

Nous pouvons également définir les coefficients de \textbf{réflexion} et de \textbf{transmission} comme suit:
\begin{align}
R &= \frac{J_- [\leftarrow]}{J_- [\rightarrow]},	&T = \frac{J_+ [\rightarrow]}{J_- [\rightarrow]}
\end{align}
Où $J_- [\leftarrow]$ est défini comme étant "le courant de probabilité dans la région négative allant vers la gauche."\\

Dans le cas d'une onde plane, nous pouvons tautologiquement réécrire \eqref{Solution directe} sous la forme
\begin{equation}
\label{One plane forme}
\Psi = Ce^{i \bm{k} \cdot \bm{x}} e^{-i \omega t}
\end{equation}
Dès lors, $\rho = \norm{\Psi}^2 = C^2$ est une constante. Nous pouvons utiliser le résultat ($\ref{Courant de probabilité}$) pour voir que, dans le cas d'une \textbf{onde plane}, 
\begin{equation}
\bm{J} = \frac{\bm{p}}{m} C^2 = \rho\bm{v}
\end{equation}
Nous voyons donc bien qu'une onde plane décrit une particule se déplaçant à une vitesse $\bm{v}$.
\subsection{Born: étude des collisions. Origine de l'interprétation probabiliste}
Que se passe-t-il si une onde plane arrive sur un atome? Nous allons essayer de trouver une approximation à cette question.\\

Une fonction d'onde dont l'énergie potentielle V($\bm{r}$) ne dépend pas du temps doit vérifier l'équation de Schrödinger:
\begin{equation}
i \h \partial_t \Psi = - \frac{\h^2}{2m} \Delta \Psi + V(\bm{r}) \Psi
\end{equation}
On parle alors de \textit{potentiel stationnaire}. La solution de cette équation est donnée par
\begin{equation}
\Psi (\bm{r},t) = e^{- \frac{iEt}{\h}} \phi(\bm{r}) - \frac{\h^2}{2m} \Delta \phi + V(\bm{r})\phi = E \phi
\end{equation}
\textbf{Conditions au bord : } pour $x\rightarrow -\infty, \phi = Ce^{ikx}$.
\begin{itemize}
\item Une partie de l'onde est construite tout droite.
\item Une partie est diffusée.
\end{itemize}
A grande distance de l'atome, nous avons que 
\begin{equation}
\phi(\bm{r}) \approx Ce^{ikx} + \int d^3k \alpha(\bm{r})e^{i \bm{k} \cdot \bm{x}}
\end{equation}
\begin{center}
Avec $\frac{\h^2\norm{\bm{k}}^2}{2m} = \h \omega$.
\end{center}
La Mécanique Quantique décrit simultanément \textbf{toutes} les diffusions possibles. Or, en laboratoire, nous n'observons qu'une seule direction: Max-Born en déduit que la Mécanique Quantique décrit les \textbf{possibilités} de diffusions dans les directions $\bm{k}$.
\begin{equation}
P(\text{diffusion dans la direction $\bm{k}$}) \approx \norm{\alpha(\bm{k})}^2
\end{equation}

\section{Paquet d'onde à une dimension}
\subsection{Vitesse de phase et vitesse de groupe}
Une onde est une perturbation se déplacnt dans le milieu. Il est possible de lui associer deux vitesses: soit la \emph{vitesse de phase} (ou \emph{célérité}), et la \emph{vitesse de groupe}. Elle peuvent être différentes, sous certaines conditions.\\

Revenons au cas particulier d'une particule libre, dont l'état est décrit par le paquet d'onde à une dimension ($\ref{Solution générale}$). Une onde plane ($\ref{One plane forme}$), respectant la relation de dispersion $w(k)$, se propage avec la vitesse
\begin{equation}
V_\phi (k) = \frac{\omega}{Re\{k\}}
\end{equation}
Il s'agit de la \textbf{vitesse de phase}. En paticulier, prenons l'exemple d'une particulier quantique, c'est à dire une particule respectant les équations
\begin{align}
\omega = \frac{E}{\h} &= \frac{p^2}{2m\h} = \h \frac{k^2}{2m}\\
k &= \frac{p}{\h}
\end{align}
Dans cet exemple, la vitesse de phase s'écrira alors
\begin{equation}
V_\phi (k) = \frac{\h k}{2m} = \frac{p}{2m}
\end{equation}
Nous savons que dans le cas d'une onde électromagnétique se propageant dans le vide, $V_\phi (k)$ est indépendante de $k$ et se propage à la vitesse de la lumière $c$. Notons que toutes les ondes composant un paquet d'ondes se déplacent à la même vitesse, de sorte que le paquet se déplace à la vitesse de la lumière $c$ dans le vide. \textbf{Ce n'est pas le cas dans un milieu dispersif.}\\

Soit $A(t,x) = \int dk g(k,\omega) e^{-i\omega(k)t}e^{ikx}$, où
\begin{align*}
g(k,\omega)&: \text{est centré sur $k_0$ de faible largeur $\Delta$}\\
g(k,\omega) &\approx e^{- \frac{1}{2}(\frac{k-\omega}{\Delta})^2}
\end{align*}
Nous voulons nous rammener à une intégrale gaussienne. Notons que 
\begin{equation}
\omega(k) = \omega (k_0) + \partial_k \omega (k-k_0) + \frac{1}{2}\partial_k^2(k-k_0)
\end{equation}
Le terme $\partial_k \omega$ représente la \textbf{vitesse de groupe}, et $\partial_k^2 \omega$ représente la \textbf{dispersion}.

Nous avons dès lors que
\begin{equation}
A(t,x) \approx e^{-i \omega(k_0)t}e^{ik_0x} \int dk g(k-k_0) e^{\left[\omega't+x\right]}
\end{equation}
Nous y négligeons les effets de $\partial_k^2 \omega$ car $\Delta$ est très petit.\\

Il s'ensuit que le centre du paquet d'onde se déplace à la vitesse 
\begin{equation}
v_g = \text{Vitesse de groupe} = \frac{\partial \omega}{\partial k} = \omega'
\end{equation}
Pour une particule quantique, nous avons alors que $\frac{\partial \omega}{\partial k} = \frac{\h k}{m} = \frac{p}{m} = v_{Classique}$.

\textbf{Note.} Rappelons que la solution à l'intégrale ci-contre, dans le cas où $\textbf{si} -\frac{\pi}{4} < arg \alpha < + \frac{\pi}{4}$:
\begin{equation}
I(\alpha,\beta) =\int_{-\infty}^{\infty} e^{-\alpha^2 \left(\xi+\beta\right)^2} d\xi = \frac{\sqrt{\pi}}{\alpha}
\label{Integral}
\end{equation}

\paragraph{Paquet d'onde Gaussien en dim 1}
Nous considérons un modèle à une dimension, avec une particule libre\footnote{Le potentiel V(x) = 0}, dont la fonction d'onde à l'instant $t = 0$ s'écrit
\begin{equation}
\Psi(x,t=0) = \frac{\sqrt{a}}{\left(2\pi\right)^{3/4}} \int_{-\infty}^{+\infty} e^{- \frac{a^2}{4} (k-k_0)^2} e^{ikx} dk\\
\label{Onde gaussienne dim 1}
\end{equation}
Ce paquet d'onde est obtenu par superposition d'ondes planes $e^{ikx}$ avec des coefficients 
\begin{equation}
\frac{1}{\sqrt{2\pi}} g(k,t=0) = \frac{\sqrt{a}}{\left(2\pi\right)^{3/4}} e^{-\frac{a^2}{4}(k-k_0)^2}
\end{equation}
qui correspondent à une fonction de Gauss, centrée en $k = k_0$. C'est pourquoi nous appelons \eqref{Onde gaussienne dim 1} onde gaussienne.\\

En exploitant le résultat \eqref{Integral}, nous pouvons alors montrer que \ref{Onde gaussienne dim 1} vaut
\begin{equation}
\left( \frac{2}{\pi a^2}\right)^{\frac{1}{4}} e^{ik_0x}e^{- \frac{x^2}{a^2}}
\end{equation}
Nous pouvons en déduire la valeur de la densité de probabilité de la fonction d'onde d'une particule libre à l'instant $t = 0$:
\begin{equation}
\norm{\Psi(x,t=0)}^2 = \sqrt{\frac{2}{\pi a^2}} e^{-2 \frac{x^2}{a^2}}
\end{equation}
Nous pouvons vérifier que $\int dx \norm{\Psi}^2$ est bien égale à 1.\\

%Calcul de \Psi(x,t).

Nous remarquons alors que $\Delta x \approx \frac{a}{2}$. Ainsi, $\Delta x\Delta p \geq \frac{\h}{2}$. Nous en déduisons en particulier que $\Delta p = \frac{\h}{2}$, et donc que la distribution des vitesses est donnée par $\Delta v = \frac{\Delta p}{m} = \frac{\h}{ma}$ : nous avons un \textbf{étalement}.

\newpage
\section{Potentiel stationnaire}
Nous nous intéressons ici au comportement d'une onde plongée dans un potentiel indépendant du temps ; V($\bm{r},t$) = V($\bm{r}$). Cela signifie que les effets quantique doivent se produire lorsque le potentiel varie sur des distances plus courtes que la longueur d'onde : celles-ci ne peuvent alors pas être négligées.  Nous allons donc étudier le comportement d'une particule placée dans différents "potentiels carrés", c'est à dire des potentiels dont les variations se font par "marche d'escaliers". Avant de passer à l'étude du potentiel, discutons des propriétés que satisfait l'équation de Schrödinger pour un potentiel indépendant du temps $V(\bm{r})$.\\

\subsection{Equation de Schrödinger indépendante du temps}

Recherchons les solutions de l'équation de Schrödinger \eqref{Schrodinger}. Pour ce faire, passons par une séparation des variables. Plus précisément, posons
\begin{equation}
\Psi(\bm{r},t) = \varphi(\bm{r})\kappa(t)
\end{equation}
Il suffit alors de placer cette dernière équations dans $\eqref{Schrodinger}$:
\begin{align}
i \h \varphi(\bm{r})\partial_t \kappa(t) &= - \frac{\h^2}{2m} \kappa(t) \Delta \varphi(\bm{r}) + V(\bm{r})\kappa(t)\varphi(\bm{r})\\
\frac{i\h}{\kappa(t)} \partial_t \kappa(t) &= - \frac{\h^2}{2m} \frac{1}{\varphi(\bm{r})} \Delta \varphi(\bm{r}) + V(\bm{r})
\label{Séparation temporelle et spatiale}
\end{align}
Cette équation indique l'égalité entre une fonction de t (membre de gauche) et une fonction de $\bm{r}$ (membre de droite). Cette dernière n'est possible que si ils sont en fait tous les deux égales à une constante, que nous poserons (par convention) égale à $\h \omega$.\\

Dans le membre de gauche, nous obtenons une équation différentielle non-linéaire du premier ordre. Elle se résoud trivialement en
\begin{align}
&\kappa (t) = c_1e^{-i\omega t}    &\forall c_1 \in \mathbb{R}
\label{Solution temporelle}
\end{align}
En développant l'égalité de droite, nous trouvons alors \textbf{l'équation de Schrödinger indépendante du temps}:
\begin{align}
- \frac{\h^2}{2m} \Delta \varphi(\bm{r}) + V(\bm{r})\varphi(\bm{r}) &= \h\omega\varphi(\bm{r}) \label{Schrodinger spacial} \\
\left[ - \frac{\h^2}{2m} \Delta + V(\bm{r}) \right] \varphi(\bm{r}) &= E \varphi(\bm{r}) \label{Equations aux valeurs propres}\\
H \varphi(\bm{r}) &= E \varphi(\bm{r})  \label{kikoo}
\end{align}
En posant $c_1 = 1$ dans \eqref{Solution temporelle}, nous obtenons alors la fonction
\begin{equation}
\Psi (\bm{r},t) = \varphi(\bm{r})e^{-i\omega t}
\end{equation}
qui est solution de l'équation de Schrödinger, si $\varphi(\bm{r})$ est solution de \eqref{Schrodinger spacial}. On dit que l'on a séparé les variables de \textbf{temps} et d'\textbf{espace}.\\

\textbf{Note.} Dans \eqref{Equations aux valeurs propres}, $H$ est un opérateur différentiel \emph{linéaire}. Effectivement, si $\lambda_1$ et $\lambda_2$ sont des constantes, alors 
\begin{align}
&H \left[\lambda_1 \varphi_1(\bm{r}) + \lambda_2 \varphi_2(\bm{r})\right] = \lambda_1 H \varphi_1(\bm{r}) + \lambda_2 H \varphi_2(\bm{r})							&\forall \lambda_1,\lambda_2 \in \mathbb{R}
\end{align}

\textbf{Note.} L'équation $\eqref{Equations aux valeurs propres}$ est souvent décrite comme "l'équation aux valeurs propres".

\subsection{Potentiels à une dimension: description quantitative}

Rappelons l'équation de Schrödinger, elle nous sera utile pour la suite:
\begin{equation*}
i \h \partial_t \Psi (t,\bm{r}) = - \frac{\h^2}{2m} \partial_x^2 \Psi(t,\bm{r}) + V(\bm{r})\Psi(t,\bm{r})
\end{equation*}
Où V($\bm{r}$) est constante par morceaux.\\

Nous allons considérer les problèmes de \textit{puit de potentiel}, et de\textit{ barrière de potentiel}. Une question légitime est de se demander les raisons derrière cette étude : la réponse est simple. Ils s'agissent de cas théoriques possédant des solutions analytique, et permettant d'illuster des effets quantiques importants: l'\textbf{effet tunnel}, ainsi que les \textbf{états liés}. De plus, certaines situations physiques réelles y sont très proches: pensons notamment à la barrière de Josephson (supraconducteurs), et aux points quantiques: elles consistent dès lors en une excellent approximation de ces phénomènes.\\

\subsubsection{Puit de potentiel infini à une dimension}

Cette situation correspond à un potentiel prenant les valeurs suivantes, en fonction de sa position\footnote{Insistons sur le fait que le potentiel est considéré invariant par le temps : nous avons bien un potentiel stationnaire du type $V(\bm{r},t) = V(x)$.} : 
\begin{equation}
V(x) = 
\begin{cases}
-\infty\\
0\\
+\infty
\end{cases}
\end{equation}
respectivement en $x < 0$, $x \in [0,L]$ et $x >L$. Appliquons ce potentiel à l'équation de Schrödinger.

\textbf{Séparation des variables:} $\Psi(\bm{r},t) = \kappa(t)\varphi(\bm{r})$. Nous avons un potentiel stationnaire : reprenons le calcul à partir de $\eqref{Equations aux valeurs propres}$.\\

La solution temporelle provient de $\eqref{Séparation temporelle et spatiale}$, et donne 
\begin{equation}
\kappa(t) = \kappa_0 e^{-i \omega t}
\end{equation}
Il nous reste à déterminer la solution spatiale. Rappelons les conditions aux bords:  $\varphi(L) = 0 = \varphi(0)$. Dès lors, nous avons que:\\

Pour $0 < x < L : - \frac{\h^2}{2m} \partial_x^2 \varphi = E \varphi$. Il s'agit d'une équation différentielle linéaire du second ordre. Sa solution \textit{générale} est de la forme
\begin{align}
&\varphi(\bm{r}) = c_1 \cos \left(kx\right) + c_2\sin \left( kx \right)									&\forall c_1,c_2 \in \mathbb{C}
\label{Kikoo}
\end{align}
Afin de respecter les conditions aux bords, nous aurons que
\begin{align}
&c_2 = 0 &kL = n\pi
\end{align}
Dès lors,
\begin{align}
&k_n = \frac{n\pi}{L} &E_n = \frac{\h^2}{2m} \frac{\pi^2n^2}{L^2}
\end{align}
Nous avons alors la solution mathématique à notre problème:
\begin{equation}
\Psi_n (\bm{r},t) = \kappa_0 e^{-i \frac{E_nt}{\h}}\sin \left( \frac{n\pi}{L}x \right)
\end{equation}
\textbf{Note.} Notons que cette équation implique une \textbf{quantification de l'énergie}. La solution physique sera la superposition de tous les états possible:z
\begin{equation}
\Psi (\bm{r},t) = \sum_{n = 1}^\infty \kappa_n e^{-i \frac{E_nt}{\h}}\sin \left( \frac{n\pi}{L}x \right)
\end{equation}

\subsubsection{Puit de potentiel fini à une dimension}
Etudions un cas similaire. Soit $I = \left[-\frac{a}{2},\frac{a}{2}\right]$. Dans cette situation, nous avons que 
\begin{equation}
\begin{cases} 
V(x > \norm{\frac{a}{2}}) = 0 \\ 
V(x) = -V_0\\
\end{cases}
\end{equation}

La résolution de l'équation aux valeurs propres $\eqref{Equations aux valeurs propres}$ s'effectue dans chaque zone\footnote{Nous notons I la zone tq $x < -\frac{a}{2}$, III la région tq $x>\frac{a}{2}$ et II la région entre les deux.} indépendemment, et donnera
\begin{equation}
\begin{cases}
\varphi(x < -\frac{a}{2}) = A_1e^{\rho x} + A'_1e^{-\rho x}		&\rho = \sqrt{- \frac{2mE}{\h^2}}\\
\varphi(x \in I) = B_2e^{ikx} + B'_2e^{-ikx} 					&k = \sqrt{\frac{2m(V_0+E)}{\h^2}}\\
\varphi(x>\frac{a}{2}) = A_3e^{\rho x} + A'_3e^{-\rho x}
\end{cases}
\end{equation}
\textbf{Conditions aux bords.} Rappelons que $\varphi(x)$ est bornée en les régions I et III. Nous pouvons alors réécrire les équations sous la forme
\begin{equation}
\begin{cases}
\varphi(x) = A_1e^{\rho x}\\
\varphi(x \in I) = B_2e^{ikx} + B'_2e^{-ikx}\\
\varphi(x) = A'_3e^{-\rho x}
\end{cases}
\end{equation}
\textbf{QUESTION.} Comment avons-nous choisi le terme en $A'_1$ et $A_3$ pour la simplification?\\

\textbf{Conditions de continuité:} $\varphi(x)$ et $\partial_x \varphi(x)$ doivent être continue. Dès lors,
\begin{itemize}
\item \textbf{En x = - $\frac{a}{2}$}, nous avons que:
\begin{equation}
\begin{cases}
A_1e^{-\frac{\rho a}{2}} = B_2e^{\frac{-ika}{2}} + B'_2e^{\frac{ika}{2}}\\
\rho A_1e^{-\rho\frac{a}{2}} = ik \left[B_2e^{-\frac{ika}{2}} - B'_2e^{\frac{ika}{2}} \right]
\end{cases}
\label{Continuité en -a/2}
\end{equation}
\item \textbf{En x = $\frac{a}{2}$}, nous avons que:
\begin{equation}
\begin{cases}
A'_3e^{-\frac{\rho a}{2}} = B_2e^{\frac{-ika}{2}} + B'_2e^{\frac{ika}{2}}\\
\rho A'_3e^{-\frac{\rho a}{2}} = ik \left[B_2e^{\frac{-ika}{2}} - B'_2e^{\frac{ika}{2}}\right]
\end{cases}
\label{Continuité en a/2}
\end{equation}
\end{itemize}
\textbf{Note.} Nous possédons 4 équations linéaires à 4 inconnues : la solution est non triviale si $\det \left[ \text{Matrice associée} \right] = 0$.\\
\textbf{Note.} Les inconnues dans nos équations sont bien $A_2,A_2',B_3$ et $B'_3$.\\

En multipliant la première équation de $\eqref{Continuité en -a/2}$ par $ik$, et en additionnant/soustrayant les deux équations, nous pouvons obtenir
\begin{equation}
\begin{cases}
B_2 = \frac{\rho + ik}{2ik}e^{\left(ik-\rho\right)\frac{a}{2}}A_1\\
B'_2 = -\frac{\rho-ik}{2ik} e^{-\frac{a}{2}\left[\rho+ik\right]}A_1
\end{cases}
\end{equation}
Similairement, \eqref{Continuité en a/2} permet de mettre en évidence les relations
\begin{equation}
\begin{cases}
B_2 = -\frac{\rho-ik}{2ik}e^{-\frac{a}{2}\left(\rho+ik\right)}A'_3\\
B'_2 = \frac{\rho+ik}{2ik}e^{\frac{a}{2}\left(ik-\rho\right)}A'_3
\end{cases}
\end{equation}
Ces deux dernières équations, ensembles, impliquent:\\

(...)\\

Les états d'énergie les plus sont sont alors
\begin{align}
\approx k &= \frac{n\pi}{a}		&E \approx \frac{\pi^2\h n^2}{2ma^2} - V_0
\end{align}

\subsubsection{Potentiel en escalier}
Plaçons-nous dans une région telle que le potentiel est stationnaire, c'est à dire dont la variable spatiale $\varphi (\bm{r})$ de la solution de l'équation de Schrödinger $\Psi (\bm{r},t)$ respecte l'équation $\eqref{Schrodinger spacial}$. Nous pouvons réécrire cette dernière équation sous la forme
\begin{equation*}
\frac{d^2 \varphi}{dx^2} + \frac{2m}{\h^2} (E-V)\varphi = 0
\end{equation*}
Nous pouvons distinguer plusieurs cas.
\begin{itemize}
\item \textbf{$E > V$}. En introduisant le terme positif $k^2 = \frac{2m}{\h^2} (E-V) \geq 0$, nous pouvons montrer que les solutions de $\eqref{Schrodinger spacial}$ sont de la forme
\begin{equation}
\varphi = Ae^{ikx} + A'e^{-ikx}
\end{equation}
Nous parlons d'ondes progessives.
\item \textbf{$E < V$}. Cette condition correspond aux régions classiquement interdites : il s'agit de l'\textbf{effet tunnel}. Dans ce cas, nous introduisons la constante $\rho$ définie par $\rho^2 = \frac{2m}{\h^2} (V-E) \geq 0$. Nous obtenons alors que la solution est
\begin{equation}
\varphi = Be^{\rho x} + B' e^{-\rho x}
\end{equation}
\item \textbf{E = V}. Dans ce cas, $\varphi (\bm{r})$ est une fonction linéaire de x.
\item \textbf{Là où V est discontinue}. Dans ce cas, $\varphi$ est continue et $\partial_x\varphi$ l'est également.
\end{itemize}
\section{Approximation semi-classique}
L'approximation semi-classique permet d'obtenir une solution de l'Equation de Schrödinger lorsque $\h$ tend vers 0 ; c'est à dire lorsque la longueur d'onde est beaucoup plus ptite que les autres dimensions considérées. Son idée est simple: l'équation de Schrödinger se dérive de l'équation de propagation des ondes. On doit alors retrouver la mécanique classique dans la limite $\h$ tend vers 0, tout comme nous retrouvons l'optique géométrique lorsque $\lambda$ tend vers 0 dans l'optique ondulatoire.\\

Notons $\varphi$ solution stationnaire de l'équation de Schrödinger pour une particule de masse $m$ dans un potentiel $V(R)$. Alors,
\begin{equation*}
\left[ -\frac{\h^2}{2m} \partial_x^2 + V(R)\right]\Psi(r) = E\Psi(r)
\end{equation*}
Se réécrit, en posant 
\begin{align}
&\Psi(r) = A(r)e^{i\frac{S(r)}{\h}}	&\forall A,S\in\mathbb{R}
\end{align}
Nous pouvons alors montrer que les relations
\begin{subequations}
\begin{equation}
2A'S'+AS'' = 0 \label{3}\\
\end{equation}
\begin{equation}
\frac{S'^2}{2m} - \frac{\h^2}{2m}\frac{A''}{A}+V = E \label{4}
\end{equation}
\end{subequations}

sont équivalentes à l'équation de Schrödinger \eqref{Schrodinger spacial}. En particulier, \eqref{3} peut se résoudre directement et donne
\begin{align}
&A(x) = \frac{A_0}{\sqrt{S'(r)}}	&\forall A_0\in\mathbb{R}
\label{A}
\end{align}
Nous pouvons vérifier que \eqref{A} est équivalente à l'équation de continuité \eqref{Equation de Continuité}
\begin{equation*}
\partial_t\rho(\bm{r},t) + \nabla \bm{J} = 0
\end{equation*}
pour une solution stationnaire: effectivement, nous avons que $\rho (x,t) = |\psi|^2$ ne dépend pas du temps.\\

Pour résoudre \eqref{4}, nous faisons l'hypothèse $\frac{\h^2}{2m}\frac{A''}{A}$ est négligeable par rapport aux autres termes. Nous obtenons alors l'équation 
\begin{align*}
\frac{S'^2 (r)}{2m} + V(r) = E
\end{align*}
Il s'agit d'une équation bien connue de la mécanique classique: l'équation de \textbf{Hamilton-Jacobi}. Ses solutions sont de la forme
\begin{subequations}
\begin{align}
S'(x) &= \pm \rho (r)		&\rho (r) = \sqrt{2m(E-V(r))}\\
S(x) &= \pm \int^x dx' \rho (x')
\end{align}
\end{subequations}
Nous avons alors que
\begin{equation}
\Psi (r) = \frac{A_0}{\sqrt{\rho (r)}}e^{\pm i \int^x dx' \frac{\rho (x')}{\h}}
\end{equation}
Nous pouvons en déduire que
\begin{enumerate}
\item Le nombre d'onde à la position x est donnée par $\bm{k}(x) = \frac{p(x)}{\h}$.
\item La longueur d'onde à la position x est donnée par $\lambda (x) = \frac{2\pi\h}{p(x)}$.
\item La vitesse de groupe est donnée par
\begin{equation}
\frac{1}{V_g} = \frac{\partial k(r)}{\partial \omega} = \frac{\partial p(r)}{\partial E} = \frac{m}{p(r)} = \frac{1}{V_{classique} (r)}.
\end{equation}
La vitesse d'un paquet d'onde sera donnée, selon l'approximation semi-classique, par la vitesse $V_{classique} (r)$ de la mécanique classique.
\item Dans une région classiquement interdite,
\begin{align}
&\Psi (r) = \frac{1}{\sqrt{\rho (r)}} e^{\pm \int^x dx' \frac{\rho(x')}{\h}}	&\text{Avec } \frac{\rho^2 (x)}{2m} = V(x)-E
\end{align}
\end{enumerate}

[\textbf{Graphique}]

Nous pouvons montrer que si la solution décroit exponentionellement à grande distance (proche du point de rebroussement):
\begin{align}
\Psi(r) &= \frac{1}{\sqrt{k(x)}} \cos (\int_b^x k(x') dx' - \frac{\pi}{4})	&\forall x > b\\
\Psi(r) &= \frac{1}{\sqrt{k(x)}} \cos (\int_x^a k(x')dx'+\frac{\pi}{4})		&\forall x < a
\end{align}
Nous obtenons la condition de quantification semi-classique:
\begin{align}
\frac{1}{\h} \int^a_b dx\sqrt{2m(E-V(x)} &= (n+\frac{1}{2})\pi
\end{align}
(Valable uniquement pour E grand).

\subsection{Application à la désintégration alpha des noyaux}

La particule alpha a une énergie E. Pour $R>R_\alpha$, nous avons une énergie $V(R)<E$ et sommes alors dans une région classiquement permise. Malheureusement, la particule va devoir traverser une région classiquement interdite - entre $R$ et $R_\alpha$. La probabilité d'émission par unité de temps est approximée par $\frac{1}{T_{\frac{1}{2}}} \sim \frac{1}{e^{2\gamma}}$, où:
\begin{align}
\gamma &= \frac{1}{\h} \int_R^{R_\gamma} dr\sqrt{2m_\alpha (V(r)-E)}		&\text{Avec } V(r) = \frac{z_\alpha ze^2}{4\pi\epsilon_0r} \text{ et } E = \frac{z_\alpha ze^2}{4\pi\epsilon_0R_\alpha}\\
&= \frac{1}{\h} \sqrt{\frac{z_\alpha ze^2}{4\pi\epsilon_0}} \int_R^{R_\alpha} \sqrt{\frac{1}{r} - \frac{1}{R_\alpha}} dr\\
\gamma &\approx \frac{\pi}{2\h} \sqrt{2m_\alpha} \left[\frac{z_\alpha ze^2}{4\pi\epsilon_0}\right]\frac{1}{\sqrt{E}}\\
&\rightarrow \log T_{\frac{1}{2}} = a\frac{z}{\sqrt{E}} + b
\end{align}
Cette loi est bien vérifiée expérimentalement. Elle explique pourquoi il n'y a pas de désintégration des noyaux les plus lourds.
\end{document}
