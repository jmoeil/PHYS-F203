\documentclass[../../notesdecours]{subfiles}

\begin{document}
Empiquement, nous trouvons que l'énergie d'un atome d'hydrogène sera donné par la relation
\begin{equation}
\label{Energie hydrogène}
E = \frac{p^2}{2m} - \frac{e^2}{r}
\end{equation}
Où $e^2 = \frac{q_e^2}{4 \pi \epsilon_0}$.\\

Or, nous avons que l'électron de l'atome sera confiné dans une zone de rayon r. Dès lors, en utilisant $\ref{Heinsenberg}$, nous aurons que $\Delta p \approx \frac{\hbar}{r}$. Dès lors,
\begin{equation}
E \approx \frac{\hbar^2}{2mr^2} - \frac{e^2}{r}
\end{equation}
Il suffit alors de dériver cette dernière pour obtenir l'énergie minimale:
\begin{align*}
E_{min} &= \frac{d}{dr} (\frac{\hbar^2}{2mr^2} - \frac{e^2}{r}) = 0\\
&= - \frac{2 \h^2}{2mr^3} + \frac{e^2}{r^2} = 0\\
&= \frac{1}{r^2} \left[e^2 - \frac{\h^2}{mr} \right] = 0
\end{align*}
Il s'ensuit dès lors que 
\begin{center}
\begin{tabular}{c c}
$r = \frac{\hbar^2}{me^2}$ & $E_{min} = - \frac{me^4}{2 \hbar^2}$
\end{tabular}
\end{center}
Nous pouvons en déduire la valeur du $\textbf{rayon de Bohr}$ - soit la distance séparant, dans l'atome d'hydrogène, le proton de l'électron. Il s'agit donc d'un ordre de grandeur du rayon des atomes. Il correspond à 
\begin{equation}
\label{Rayon de Bohr}
a_0 = \frac{\hbar^2}{me^2}
\end{equation}
Similairement, nous avons l'énergie de liaison de l'atome d'Hydrogène - également appelée \textbf{énergie de Rydberg} :
\begin{equation}
R_y = \frac{me^4}{2 \hbar^2}
\end{equation}
Nous pouvons retrouver les états liés en suivant
\begin{equation}
\label{Etats liés}
E_n = - R_y \frac{1}{n^2} = - \frac{me^4}{2 \hbar^2} \frac{1}{n^2}
\end{equation}
Où n = 1,2,3, ...
\end{document}