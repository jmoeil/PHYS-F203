\documentclass[../Notes de cours]{subfiles}

\begin{document}
\part{Principe d'incertitude d'Heisenberg}
Empiriquement, il est impossible de déterminer à la fois la position $x$ et l'impulsion $p$ d'une particule au delà d'une certaine précision

\begin{equation}
\label{Heinsenberg}
\Delta x \Delta p \geq \frac{\hbar}{2}
\end{equation}

où $\Delta x$,$\Delta p$ sont les \textbf{écart-types} des grandeurs liées. Il s'agit de l'Incertitude de Heisenberg.\\

Nous avons donc que pour tout état quantique, $x$ et $p$ sont incertains : leurs incertitude obéissent à la relation $\ref{Heinsenberg}$. Cela est liée au caractère probabiliste de la Mécanique Quantique : chaque résultat d'une mesure est aléatoire. Cela fait l'objet de la discussion \ref{Postulat3}.\\

Notons la relation de longueur d'onde de Broglie : 
\begin{equation}
\label{Broglie}
\lambda = \frac{h}{p}
\end{equation}

\section{Applications du Principe d'Incertitude}
Les différents calculs qui vont suivre servent à déterminer un ordre de grandeur : nous ne prêtons pas attention aux constantes multiplicatives tel que le facteur $\frac{1}{2}$ devant $\ref{Heinsenberg}$.
\subsection{Application à l'atome d'Hydrogène}
Empiquement, nous trouvons que l'énergie d'un atome d'hydrogène sera donné par la relation
\begin{equation}
\label{Energie hydrogène}
E = \frac{p^2}{2m} - \frac{e^2}{r}
\end{equation}
Où $e^2 = \frac{q_e^2}{4 \pi \epsilon_0}$.\\

Or, nous avons que l'électron de l'atome sera confiné dans une zone de rayon r. Dès lors, en utilisant $\ref{Heinsenberg}$, nous aurons que $\Delta p \approx \frac{\hbar}{r}$. Dès lors,
\begin{equation}
E \approx \frac{\hbar^2}{2mr^2} - \frac{e^2}{r}
\end{equation}
Il suffit alors de dériver cette dernière pour obtenir l'énergie minimale:
\begin{align*}
E_{min} &= \frac{d}{dr} (\frac{\hbar^2}{2mr^2} - \frac{e^2}{r}) = 0\\
&= - \frac{2 \h^2}{2mr^3} + \frac{e^2}{r^2} = 0\\
&= \frac{1}{r^2} \left[e^2 - \frac{\h^2}{mr} \right] = 0
\end{align*}
Il s'ensuit dès lors que 
\begin{center}
\begin{tabular}{c c}
$r = \frac{\hbar^2}{me^2}$ & $E_{min} = - \frac{me^4}{2 \hbar^2}$
\end{tabular}
\end{center}
Nous pouvons en déduire la valeur du $\textbf{rayon de Bohr}$ - soit la distance séparant, dans l'atome d'hydrogène, le proton de l'électron. Il s'agit donc d'un ordre de grandeur du rayon des atomes. Il correspond à 
\begin{equation}
\label{Rayon de Bohr}
a_0 = \frac{\hbar^2}{me^2}
\end{equation}
Similairement, nous avons l'énergie de liaison de l'atome d'Hydrogène - également appelée \textbf{énergie de Rydberg} :
\begin{equation}
R_y = \frac{me^4}{2 \hbar^2}
\end{equation}
Nous pouvons retrouver les états liés en suivant
\begin{equation}
\label{Etats liés}
E_n = - R_y \frac{1}{n^2} = - \frac{me^4}{2 \hbar^2} \frac{1}{n^2}
\end{equation}
Où n = 1,2,3, ...

\subsection{Application à l'Oscillateur Harmonique}
\label{Application à l'Oscillateur Harmonique}
Soit $E = \frac{p^2}{2m} + \frac{kx^2}{2}$. Posons $w = \sqrt{\frac{k}{m}}$, la fréquence angulaire. Par définition, l'état fondamental est de longueur $\Delta x$. Dès lors, $\ref{Heinsenberg}$ nous implique que l'impulsion est donnée par $\Delta p = \frac{\hbar}{\Delta x}$. En partant de là,
\begin{equation}
E \left(\Delta x \right) = \frac{\hbar^2}{2m\Delta x^2} + \frac{1}{2} k \Delta x^2
\end{equation}
En dérivant cette dernière équations, nous avons que le minimum d'énergie est donnée en $\Delta x \approx \frac{\hbar^{\frac{1}{2}}}{\left(mk\right)^{\frac{1}{4}}}$. Cela implique que
\begin{equation}
\label{Energie OH}
E_{min} \approx \sqrt{\frac{k}{m}} \hbar \approx w \hbar
\end{equation}

La relation exacte correspond effectivement. En effet :
\begin{align}
\label{Energie quantifiée OH}
E_n &= \hbar w \left( n + \frac{1}{2} \right)\\
E_0 &= \frac{1}{2} \hbar w
\end{align}
Notons que cette dernière est ce que nous appelons l'\textbf{énergie de point zéro}.
\subsection{Application au cas relativiste}
L'incertitude d'Heisenberg est également valide dans le cas relativiste. La relation de dispersion nous donne que $E^2 = m^2 c^4 + p^2 c^2$ pour une particule libre relativiste. En posant $\Delta x \approx \frac{\hbar}{mc}$, nous avons que $\Delta p \approx mc$ : il s'agit d'une particule relativiste. Dès lors,
\begin{itemize}
\item L'incertitude sur l'énergie sera de l'ordre $\Delta E \approx mc^2$.
\item Supposons que nous avons un électron $e^(-)$ dans une boite de taille $\approx \frac{\hbar}{mc}$. L'incertitude sur l'énergie permet alors de créer des paires d'électrons et de positrons. \textbf{La notion de particule perd son sens dans le cadre de la mécanique quantique relativiste}. Cette taille est appelée à longueur d'onde de Compton, et vaut exactement $\lambda_C = \frac{\hbar}{mc}$.
\end{itemize}

\subsection{Application à la Masse de Planck}
Considérons une particule de masse M confinée dans une boule de rayon R.
%Schéma
Nous avons alors plusieurs longueurs charactéristiques intéressantes.
\begin{enumerate}
\item \textbf{Première longueur charactéristique.} La longueur de Compton : $\lambda_C = \frac{\hbar}{Mc}$.
\item \textbf{Seconde longueur charactéristique.} Le rayon de Schwarzschild : $R_S = \frac{2GM}{c^2}$. Il s'agit du rayon que doit prendre un objet de masse M pour devenir un trou noir ; c'est à dire dont la vitesse de libération est de l'ordre c.
\end{enumerate}

La masse de Planck est alors donnée par la relation $M_{Pl} = \sqrt{\frac{\hbar c}{G}}$. En particulier, nous avons que :
\begin{itemize}
\item Si M $< M_{Pl}$, alors $\lambda_C > R_S$ : nous avons une particule élémentaire.
\item Si M $> M_{Pl}$, nous aurons un trou noir.
\item Si M $\approx M_{Pl}$, nous ne connaissons pas la nature de l'objet.
\end{itemize}

\subsection{Application à la masse des étoiles}
Soit N le nombre d'atomes d'hydrogène dans une boule de rayon R, soumis à l'attraction gravitationelle. Le volume par atome est de l'ordre $\frac{R^3}{N}$. A partir de là, nous pouvons déduire:
\begin{itemize}
\item \textbf{Le rayon de confinement de chaque électron et proton.} $\Delta x_e = \Delta x_p = \frac{R}{N^{\frac{1}{3}}}$.
\item \textbf{Le moment de ces mêmes électrons et protons\footnote{A partir de \ref{Heinsenberg}.}.} $\Delta p_e = \Delta p_p = \frac{\hbar N^{\frac{1}{3}}}{R} $.\\
\end{itemize}

De ces relations, nous pouvons écrire les énergives cinétiques et gravitationelle de l'objet : 
\begin{center}
\begin{tabular}{c c}
Energie cinétique & Energie gravitationelle\\
$N \left( m_e c^2 + \frac{1}{2} \frac{\Delta P_e^2}{m_e} + m_p c^2 + \frac{1}{2} \frac{\Delta P_p^2}{m_p} \right)$ & $- G \frac{(Nm_p)^2}{R}$
\end{tabular}
\end{center}
Notons que nous négligeons le dernier terme de l'énergie cinétique : de fait, $m_p >> m_e$.
Dès lors, l'énergie sera donnée par 
\begin{equation}
E(R) \approx - G \frac{(Nm_p)^2}{R} + Nm_e c^2 + Nm_p c^2 + \frac{N}{2m_e} \frac{\hbar^2 N^{\frac{2}{3}}}{R^2}
\end{equation}
Dès lors, l'énergie sera minimum en $R^* \approx \frac{\hbar^2}{Gm_p^2m_eN^{\frac{1}{3}}}$. Remarquons que lorsque le nombre de particles N augmente, $R^*$ diminue.\\

Quand est-ce que les électrons deviennent relativistes? Remarquons que $\Delta p_e \approx cm_e = \frac{\hbar N^{\frac{1}{3}}}{R^*} = \frac{Gm_em_p^2N^{2/3}}{\hbar}$. Nous pouvons en déduire que 
\begin{equation}
N = \left( \frac{M_{Pl}}{m_p} \right)^3
\end{equation}
Lorsque les particules deviennent relativistes, les réactions nucléaires deviennent possibles. Nous avons donc de la fusion nucléaire, ce qui donne une étoile!\\

Dans une première approximation, nous aurons alors que la masse d'une étoile est donnée par 
\begin{equation}
M \approx m_p N \approx m_p \left( \frac{M_{Pl}}{m_p} \right)^3
\end{equation}
Donner les ordres de grandeur de ces nombres.
\begin{center}
\begin{tabular}{c c}
Masse de Planck & Masse d'un proton\\
$M_{Pl} \approx 10^{19} \frac{GeV}{c^2}$ & $m_p \approx 1 \frac{GeV}{c^2}$
\end{tabular}
\end{center}
Dès lors, $ M_\odot \approx 10^{57} \frac{GeV}{c^2}$. La valeur exacte est de $1.04 10^{57} \frac{GeV}{c^2}$.\\

Nous estimons les plus petites étoiles à $M \approx 0.08 M_\odot$, et les plus grandes à $M \approx 100 M_\odot$.


\subsection{Masse de Chandrasekhar}
Reprenons $E(R)$ pour $N$ atomes d'hydrogènes dans une boule de rayon R, à température nulle et en tenant compte les effets relativistes.
\begin{equation}
E(R) \approx - \frac{GN^2 m_p^2}{R} + N m_p c^2 + N \sqrt{m_e^2c^4 + \frac{\h^2N^{2/3}}{R^2}c^2}
\end{equation}
Soit $N^*$ une valeur limite. Lorsque $N = N^*$:
\begin{align}
\lim_{R\to 0} E(R) &= 0\\
\lim_{R\to 0} - \frac{GN^2 m_p^2}{R} + \frac{N^{\frac{4}{3}}\h c}{R} &= 0\\
N^* &= \left( \frac{\h c}{G} \frac{1}{m_p^3} \right) = \frac{M_{Pl}^3}{m_p^3}
\end{align}
A tempéature nulle, si $N > \frac{M_{Pl}^3}{m_p^3} = N^*$, alors la boule ne peut pas résister à son attraction gravitationnelle : elle se collapse en un trou noir.\\

La masse limite à la masse de Chandrasekhar : elle vaut approximativement $1.4 M_\odot$. Il s'agit de l'origine des trous noirs et des supernovae.

\section*{Résumé du chapitre}

Formules.

\end{document}