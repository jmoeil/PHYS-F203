\documentclass[../Notes de cours]{subfiles}


\begin{document}
\part{Principe d'incertitude d'Heisenberg}

\section{La relation d'Incertitude}
Pour commencer ce cours, il est important de souligner le fait qu'en Mécanique Quantique (MQ), l'on ne peut plus considérer le résultat d'une mesure comme en mécanique classique. 
Une manière de représenter cela est d'introduire le principe d'incertitude de Heisenberg. 
Concrètement, ce principe nous dit que : empiriquement, il est impossible de déterminer simultanément par une mesure la position $x$ et l'impulsion $p$ d'une particule au delà d'une certaine précision limitée par : 

\begin{equation}
\label{Heinsenberg}
\Delta x \Delta p \geq \frac{\hbar}{2}
\end{equation}

où 
\begin{itemize}[label = \textbullet]
	\item $\Delta x$ et $\Delta p$ sont les \textbf{écart-types} de la mesure des grandeurs liées, $x$ et $p$
	\item $\hbar = \frac{h}{2\pi}$ avec $h = 6,626 \times 10^{-34}$ J.s où $h$ est la constante de Planck. 
\end{itemize}


Cette inéquation représente ce qu'on appelle la relation d'incertitude ou encore la relation d'Heisenberg. \\
Nous avons donc que pour tout état quantique, $x$ et $p$ sont incertains : leurs incertitude obéissent à la relation $\ref{Heinsenberg}$. \\

\bg{Remarque : les états quantiques}{Qu'entend-on par \textbf{état quantique} ? De manière plus générale, on peut définir un état, en physique, par l'ensemble des aspects qui caractérisent un système, de sorte à ce que l'on puisse prévoir les résultats d'une expérience, d'une mesure. \\

En mécanique quantique, la notion d'état reste présente : on parle alors d'\textbf{état quantique}. Cependant, cette notion s'élargit dans un monde \textbf{probabiliste}. On va donc attribuer à tout état quantique d'un corpuscule une fonction d'onde $\psi(\Vec{r},t)$, de sorte que son module au carré sera interprété comme une densité de probabilité de présence à un endroit $\Vec{r}$ à un instant $t$. Nous reviendrons sur les détails plus tard. \\

Notons également la différence entre un état dans la physique classique et dans la physique quantique. L'état dans la physique classique permet de déterminer un résultat d'une mesure de manière absolue, tandis que celui dans la physique quantique permet seulement de prévoir des probabilités aux résultats d'une mesure.}

\subsection{Origine de l'incertitude sur $x$ et sur $p$}
L'incertitude sur l'impulsion et sur la position peut trouver ses intuitions physiques des raisonnements suivants :
\begin{enumerate}
    \item Cela est lié au caractère \textbf{probabiliste} de la Mécanique Quantique : chaque résultat d'une mesure est aléatoire. Cela fera l'objet de la discussion d'un postulat très important en MQ. 
    \item Cela est également lié à la longueur d'onde de de Broglie. En effet, la relation qui fait intervenir cette longueur d'onde est un résultat en conséquence du fait que l'on associe une longueur d'onde à un une particule ; c'est ce qu'on appelle la \textit{dualité onde-corpuscule}. \\
    On note cette relation par : 
    \begin{equation}
    \label{Broglie}
    \lambda = \frac{h}{p}
    \end{equation}
    On peut donc facilement voir que la longueur d'onde correspondant à une particule de masse $m$ et de vitesse $v$ est d'autant plus grande que $m$ et $v$ sont petits. Ainsi, on peut considérer l'existence d'une limite pour la masse et pour la vitesse à partir de laquelle la longueur d'onde associée devient négligeable, c'est-à-dire une limite à partir de laquelle nous nous retrouvons en physique classique. Autrement dit, les propriétés ondulatoires de la matière est impossible à mettre en évidence dans le domaine macroscopique.  \\
\end{enumerate}


\bg{Remarque : relations de De Broglie}{
\begin{itemize}[label=\textbullet]
    \item $E = h \nu$ (Energie d'un photon)
    \item $\lambda = \frac{h}{p}$ (Longueur d'onde de de Broglie)
\end{itemize}
Ces deux relations restent valables dans le cas relativiste. \\
Rappelons rapidement l'expression de l'énergie d'une particule, de masse au repos $m_0$, d'impulsion $p$, dans le cas relativiste : $E = \sqrt{p^2c^2 + m_0^2c^4}$}

\section{Applications du Principe d'Incertitude}

Nous allons voir ici quelques différentes applications du principe que nous venons de voir.\\
Notons également que ce qui importe ici sera de déterminer des ordres de grandeurs plutôt que des valeurs précises. Nous pouvons dès lors ne pas prêter attention au facteur $\frac{1}{2}$ qui apparaît dans la relation d'Heisenberg $\ref{Heinsenberg}$.

\subsection{L'atome d'Hydrogène}
Le but ici est de comprendre la stabilité des atomes et de retrouver l'ordre de grandeur de l'énergie de l'atome d'hydrogène dans son état fondamental. \\

En effet, l'énergie mécanique totale d'un atome d'hydrogène peut s'écrire comme la somme de son énergie cinétique (mouvement de l'électron) et de son énergie potentiel (coulombien) : 
\begin{align}
\label{Energie hydrogène}
E &= \frac{p^2}{2m} + \frac{q^2}{4\pi \epsilon_0 r} \notag \\
&= \frac{p^2}{2m} - \frac{e^2}{4 \pi \epsilon_0 r} \notag \\
\implies E &= \frac{p^2}{2m} - \frac{\mathcal{E}^2}{r}
\end{align}
où $\mathcal{E}^2 = \frac{e^2}{4 \pi \epsilon_0}$.\\

Or, on peut considérer que l'électron de l'atome est confiné dans une zone de rayon $r$ du à l'attraction coulombienne. 
Dès lors, en utilisant la relation d'Heisenberg $\ref{Heinsenberg}$, nous avons que $\Delta p \approx \frac{\hbar}{r}$. Ainsi : 
\begin{equation}
\label{Energie hydrogene approx}
E \approx \frac{\hbar^2}{2mr^2} - \frac{\mathcal{E}^2}{r}
\end{equation}
\paragraph{} A présent, déterminons les valeurs de $r$ pour lesquelles l'énergie est minimale. Pour ce faire, calculons les racines de la dérivée de l'énergie :
\begin{align}
\label{rayon d'energie minimale}
\frac{dE}{dr} &= \frac{d}{dr} \left( \frac{\hbar^2}{2mr^2} - \frac{\mathcal{E}^2}{r} \right) \notag \\
&= - \frac{2\hbar^2}{2mr^3} + \frac{\mathcal{E^2}}{r^2} \notag \\
&= \frac{1}{r^2} \left( \mathcal{E}^2 - \frac{\hbar^2}{mr} \right) = 0 \notag \\
\iff r&= \frac{\hbar^2}{m\mathcal{E}^2}
\end{align}

\paragraph{} Nous avons donc trouvé un rayon $r$ pour lequel l'énergie de l'atome d'Hydrogène est minimale, et cette énergie minimale vaut alors, en combinant \ref{rayon d'energie minimale} et \ref{Energie hydrogene approx}: \begin{equation}
    \label{Energie minimale}
    E_{min} = - \frac{m\mathcal{E}^4}{2 \hbar^2}
\end{equation}

En fait, le rayon $r$ donné par la relation \ref{rayon d'energie minimale} et l'énergie minimale qui lui est associée en \ref{Energie minimale} correspondent respectivement au rayon du modèle de Bohr $a_0$ (où $a_0 = \frac{\hbar^2}{m\mathcal{E}^2} \equiv$ la distance séparant, dans l'atome d'hydrogène, le proton de l'électron) et à l'énergie de Rydberg $R_y$ (où $R_y = \frac{m\mathcal{E}^4}{2\hbar^2} \equiv$ l'énergie de liaison de l'atome d'Hydrogène). \\

Plus particulièrement, il est possible de trouver l'énergie des états liés, qui est en fait un multiple de l'énergie de Rydberg :
\begin{equation}
\label{Etats liés}
E_n = - R_y \frac{1}{n^2}
\end{equation}
où $n$ = 1,2,3, ...

\bg{Remarque : les états liés}{Nous pouvons brièvement définir un état lié comme un état étant piégé entre le minimum et le maximum d'un potentiel, mais nous verrons cela plus tard dans les problèmes à une dimension.}

\subsection{Application à l'Oscillateur Harmonique (Quantique) }
\label{Application à l'Oscillateur Harmonique Quantique}

Rappelons avant tout qu'en général, on définit un \textbf{oscillateur} comme un système qui évolue périodiquement dans le temps, et l'on dit qu'il est en plus harmonique lorsque cette évolution est une fonction sinusoïdale\footnote{de fréquence et d'amplitude constantes.}. \\
Nous allons à nouveau faire appel au principe d'incertitude afin de déterminer l'énergie de l'oscillateur harmonique. 

\bg{Remarque : l'importance de l'oscillateur harmonique quantique}{Il est important de savoir que l'oscillateur harmonique joue un rôle fondamentale en physique ; il est à la base d'un très grand nombre d'applications et de théorie. Gardez cela bien en tête car nous y reviendrons plus tard dans le cours, et vous le rencontrerez certainement plusieurs fois dans le reste de votre parcours en physique! {\color{red}{trop mignon tu parles aux futurs étudiants}}}

Un petit exemple d'application parmi tant d'autres en MQ serait d'utiliser l'oscillateur harmonique pour décrire le mouvement oscillatoire d'une particule autour de sa position d'équilibre. Nous allons alors procéder comme précédemment ; écrivons l'énergie mécanique totale, de manière classique, de l'oscillateur harmonique : \begin{equation}
    \label{Energie OHC}
    E = \frac{p^2}{2m} + \frac{kx^2}{2}
\end{equation}
Posons $w = \sqrt{k/m}$, la fréquence angulaire. 
\newline Notons $\Delta x \equiv$ l'étendue de l'oscillateur harmonique dans son état fondamental. 
\newline Utilisant la relation d'Heisenberg, on trouve dès lors que l'impulsion est donnée par $\Delta p \approx \hbar/\Delta x$
\newline En partant de ces hypothèses, on trouve que l'énergie de l'état fondamental est donnée par : 
\begin{equation}
E \left(\Delta x \right) = \frac{\hbar^2}{2m\Delta x^2} + \frac{1}{2} k \Delta x^2
\end{equation}
Ensuite, nous pouvons trouver, en déterminant les racines de la dérivée de cette dernière équation, que l'énergie est minimale en $$\Delta x \approx \frac{\hbar^{1/2}}{\left(mk\right)^{1/4}}\; .$$ 
En réinjectant l'expression de l'étendue $\Delta x$ pour laquelle l'énergie est minimale, dans l'expression de l'énergie de l'oscillateur même, on trouve ainsi l'énergie minimale : 
\begin{align}
\label{Energie OH}
E_{min} &\approx \hbar \sqrt{\frac{k}{m}} \notag \\
& \approx \hbar \omega 
\end{align}

\paragraph{} De cette expression, nous pouvons en tirer une conclusion très importante qui est que l'énergie à l'état fondamental de l'oscillateur harmonique est non nul, contrairement aux résultats prédits par la physique classique. 

\paragraph{} Lorsque nous aurons les outils nécessaires pour aborder l'oscillateur harmonique quantique (OHQ) en détail, nous verrons que les valeurs que peuvent prendre l'énergie de l'OHQ sont quantifiés et vérifient la relation suivante :
\begin{align}
\label{Energie quantifiée OHQ}
E_n = \hbar \omega \left( n + \frac{1}{2} \right) \mbox{ ; $n \in \mathbb{N} $}
\end{align}
De cette expression \ref{Energie quantifiée OHQ}, nous observons que : 
\begin{itemize}[label=\textbullet]
    \item D'une part, le spectre des énergies est discrets ; ainsi une certaine quantité d'énergie définie doit être fournie au système afin de passer d'un état $n$ à un état supérieur (et à l'inverse, le système doit "perdre" une certaine quantité d'énergie pour passer à un état inférieur) ; 
    \item D'autre part, l'énergie fondamentale, pour $n=0$, vaut $$ E_0 = \frac{1}{2} \hbar \omega $$
    On l'appelle également \textbf{Energie de point zéro} (venant de l'anglais, "\textit{zero point energy}", et est bien non nulle. 
    \newline L'expression que l'on a obtenu en utilisant le principe d'incertitude concorde donc bien, en terme d'ordre de grandeur (puisque l'on avait la même expression à un facteur $\frac{1}{2}$ près seulement). 
\end{itemize} 

\subsection{Principe d'incertitude dans le cas relativiste}
\paragraph{} Il se trouve que la relation d'Heisenberg est encore valable dans un cadre relativiste ; or nous savons que dans ce cadre, l'impulsion d'une particule libre relativiste est exprimée dans la relation : 
\begin{equation}
    \label{Energie relativiste particule libre}
    E^2 = m^2 c^4 + p^2 c^2
\end{equation}
\newline Supposons en plus que l'on "confine" un électron dans une boîte de taille $\approx \frac{\hbar}{mc}$, de sorte que l'incertitude sur sa position $\Delta x \approx \frac{\hbar}{mc}$ ; autrement dit, cela revient à vouloir déterminer la position de l'électron de manière très précise (en considérant que $\frac{\hbar}{mc}$ est très petit).
\newline Ainsi, le principe d'Heisenberg contraint l'incertitude de l'impulsion à être plus grande ou égale à environ $mc$ ; $\Delta p \geq mc$. 
\newline En effet, nous pouvons alors remarquer que : 
\begin{itemize}[label=\textbullet]
\item Cela autorise la particule à être relativiste 
\item En utilisant l'équation \ref{Energie relativiste particule libre}, l'incertitude sur l'énergie vérifie alors la relation 
\newline $ \Delta E \geq mc^2$. Dans le cas où elle excède $mc^2$, il y a suffisamment d'énergie que pour donner naissance à une paire : l'électron avec son positron associé. C'est ainsi que la notion de \textit{"une particule"} perd son sens dans la mécanique quantique relativiste. On ne peut pas parler d'"une particule" car l'incertitude d'Heisenberg fait qu'il peut y avoir une paire créée à tout moment (car l'énergie est incertaine au point de créer une particule identique).
\end{itemize}

\paragraph{} Fixons à présent $\Delta p = mc$, de sorte que, par la relation d'Heisenberg, on a que $\Delta x \geq \frac{\hbar}{2mc}$. 
\newline Alors une grandeur qui caractérise cette limitation à la mesure de la position d'une particule, lorsque l'on combine mécanique quantique à la relativité restreinte, est ce qu'on appelle la longueur d'onde de Compton $\lambda_C$. Cette longueur se définit comme : \begin{equation}
    \label{longueur d'onde de compton}
    \lambda_C = \frac{h}{mc}
\end{equation}
\newline Ainsi, cela signifie qu'en terme d'ordre de grandeur, l'incertitude de la position est minorée par longueur de Compton $\lambda_C$. 

\subsection{Application à la Masse de Planck}
Considérons une particule de masse M confinée dans une boule de rayon R.
%Schéma
Nous avons alors plusieurs longueurs charactéristiques intéressantes.
\begin{enumerate}
\item \textbf{Première longueur charactéristique.} La longueur de Compton : $\lambda_C = \frac{\hbar}{Mc}$.
\item \textbf{Seconde longueur charactéristique.} Le rayon de Schwarzschild : $R_S = \frac{2GM}{c^2}$. Il s'agit du rayon que doit prendre un objet de masse M pour devenir un trou noir ; c'est à dire dont la vitesse de libération est de l'ordre c.
\end{enumerate}

La masse de Planck est alors donnée par la relation $M_{Pl} = \sqrt{\frac{\hbar c}{G}}$. En particulier, nous avons que :
\begin{itemize}
\item Si M $< M_{Pl}$, alors $\lambda_C > R_S$ : nous avons une particule élémentaire.
\item Si M $> M_{Pl}$, nous aurons un trou noir.
\item Si M $\approx M_{Pl}$, nous ne connaissons pas la nature de l'objet.
\end{itemize}

\subsection{Application à la masse des étoiles}
Soit N le nombre d'atomes d'hydrogène dans une boule de rayon R, soumis à l'attraction gravitationelle. Le volume par atome est de l'ordre $\frac{R^3}{N}$. A partir de là, nous pouvons déduire:
\begin{itemize}
\item \textbf{Le rayon de confinement de chaque électron et proton.} $\Delta x_e = \Delta x_p = \frac{R}{N^{\frac{1}{3}}}$.
\item \textbf{Le moment de ces mêmes électrons et protons\footnote{A partir de \ref{Heinsenberg}.}.} $\Delta p_e = \Delta p_p = \frac{\hbar N^{\frac{1}{3}}}{R} $.\\
\end{itemize}

De ces relations, nous pouvons écrire les énergives cinétiques et gravitationelle de l'objet : 
\begin{center}
\begin{tabular}{c c}
Energie cinétique & Energie gravitationelle\\
$N \left( m_e c^2 + \frac{1}{2} \frac{\Delta P_e^2}{m_e} + m_p c^2 + \frac{1}{2} \frac{\Delta P_p^2}{m_p} \right)$ & $- G \frac{(Nm_p)^2}{R}$
\end{tabular}
\end{center}
Notons que nous négligeons le dernier terme de l'énergie cinétique : de fait, $m_p >> m_e$.
Dès lors, l'énergie sera donnée par 
\begin{equation}
E(R) \approx - G \frac{(Nm_p)^2}{R} + Nm_e c^2 + Nm_p c^2 + \frac{N}{2m_e} \frac{\hbar^2 N^{\frac{2}{3}}}{R^2}
\end{equation}
Dès lors, l'énergie sera minimum en $R^* \approx \frac{\hbar^2}{Gm_p^2m_eN^{\frac{1}{3}}}$. Remarquons que lorsque le nombre de particles N augmente, $R^*$ diminue.\\

Quand est-ce que les électrons deviennent relativistes? Remarquons que $\Delta p_e \approx cm_e = \frac{\hbar N^{\frac{1}{3}}}{R^*} = \frac{Gm_em_p^2N^{2/3}}{\hbar}$. Nous pouvons en déduire que 
\begin{equation}
N = \left( \frac{M_{Pl}}{m_p} \right)^3
\end{equation}
Lorsque les particules deviennent relativistes, les réactions nucléaires deviennent possibles. Nous avons donc de la fusion nucléaire, ce qui donne une étoile!\\

Dans une première approximation, nous aurons alors que la masse d'une étoile est donnée par 
\begin{equation}
M \approx m_p N \approx m_p \left( \frac{M_{Pl}}{m_p} \right)^3
\end{equation}
Donner les ordres de grandeur de ces nombres.
\begin{center}
\begin{tabular}{c c}
Masse de Planck & Masse d'un proton\\
$M_{Pl} \approx 10^{19} \frac{GeV}{c^2}$ & $m_p \approx 1 \frac{GeV}{c^2}$
\end{tabular}
\end{center}
Dès lors, $ M_\odot \approx 10^{57} \frac{GeV}{c^2}$. La valeur exacte est de $1.04 10^{57} \frac{GeV}{c^2}$.\\

Nous estimons les plus petites étoiles à $M \approx 0.08 M_\odot$, et les plus grandes à $M \approx 100 M_\odot$.


\subsection{Masse de Chandrasekhar}
Reprenons $E(R)$ pour $N$ atomes d'hydrogènes dans une boule de rayon R, à température nulle et en tenant compte les effets relativistes.
\begin{equation}
E(R) \approx - \frac{GN^2 m_p^2}{R} + N m_p c^2 + N \sqrt{m_e^2c^4 + \frac{\h^2N^{2/3}}{R^2}c^2}
\end{equation}
Soit $N^*$ une valeur limite. Lorsque $N = N^*$:
\begin{align}
\lim_{R\to 0} E(R) &= 0\\
\lim_{R\to 0} - \frac{GN^2 m_p^2}{R} + \frac{N^{\frac{4}{3}}\h c}{R} &= 0\\
N^* &= \left( \frac{\h c}{G} \frac{1}{m_p^3} \right) = \frac{M_{Pl}^3}{m_p^3}
\end{align}
A tempéature nulle, si $N > \frac{M_{Pl}^3}{m_p^3} = N^*$, alors la boule ne peut pas résister à son attraction gravitationnelle : elle se collapse en un trou noir.\\

La masse limite à la masse de Chandrasekhar : elle vaut approximativement $1.4 M_\odot$. Il s'agit de l'origine des trous noirs et des supernovae.

\end{document}