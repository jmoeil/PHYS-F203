\documentclass[../Notesdecours.tex]{subfiles}

\begin{document}
\chapter{Oscillateur Harmonique Quantique}
\section{De l'importance de l'Oscillateur Hamonique}
L'importance de l'Oscillateur Harmonique en Physique ne peut pas être sous-estimé. Des exemples d'applications sont légion ; prenont la $\mathcal{M}$écanique $\mathcal{C}$lassique pour l'exemple.\\

Le plus simple reste de considérer une particule de masse $m$ se déplaçant dans un potentiel central de la forme 
\begin{align}
    V(x) &= \frac{1}{2}kx^2     &\forall k\in\mathbb{R^+}
\end{align}
Dès lors, la particule effectue un mouvement oscillatoire autour du plan $x = 0$, avec une force de rappel 
\begin{equation}
    F_x = -\frac{dV}{dx} = -kx
\end{equation}
Cette situation est régie par l'équation d'un Oscillateur Harmonique, soit
\begin{equation}
    m\ddot{x} = -kx
\end{equation}
On pose alors souvent $\omega = \sqrt{\frac{k}{x}}$ ; il s'agit de la pulsation du mouvement. La solution générale de cette équation est donnée par la relation 
\begin{align}
    x(t) &= Acos \left(\omega t-\varphi\right)       &\forall A\in\mathbb{R}^+,\forall\varphi\in [0,2\pi]
\end{align}
En particulier, nous avons que l'énergie totale de la particule s'exprime par la relation
\begin{equation}
    E = T+V = \frac{1}{2}m\omega^2A^2 = \frac{p^2}{2m}+\frac{kx^2}{2} \doteq H
\end{equation}

\begin{remark}
    L'oscillateur harmonique joue un rôle fondamental en Physique ; il permet de décrire (du moins, de manière rapprochée) les mouvements d'osccilations autour d'une position d'équilibre.
\end{remark}

\begin{remark}
    En Mécanique Quantique, l'Oscillateur Harmonique est le problème exactement soluble ayant le plus d'applications.
\end{remark}

\section{L'Oscillateur Harmonique en Mécanique Quantique}
Dans les discussions quantiques, nous remplacons les grandeurs classiques $x$ et $p$ par les observables $X$ et $P$, vérifiant la relation $[X,P]=i\h$ (voir le chapitre $\ref{Chapitre 6}$ pour plus de détails et une preuve détaillée). L'Hamiltonien quantique est donc donné par 
\begin{equation*}
    H = \frac{P^2}{2m}+\frac{kX^2}{2}
\end{equation*}

Effectuons - pour facilier les notations - les transformations canoniques suivantes:
\begin{subequations}
    \begin{equation}
        \hat{X} = \sqrt{\frac{m\omega}{\h}}X
    \end{equation}
    \begin{equation}
        \hat{P} = \frac{1}{\sqrt{m\h\omega}}P
    \end{equation}
\label{opérateurs hat}
\end{subequations}
Dès lors, nous obtenons les relations suivantes.
\begin{Property}
    En vertue des conventions $\eqref{opérateurs hat}$, la relation de commutation est donnée par
    \begin{equation}
        [\hat{X},\hat{P}] = \frac{1}{\h}[X,P] = i
    \end{equation}
\end{Property}

\begin{Property}
    L'Hamiltonien est donné par $H = \h\omega\hat{H}$, où
    \begin{equation}
        \hat{H} = \frac{1}{2}\left(\hat{X}^2+\hat{P}^2\right).
        \label{Hamiltonien quantique hat}
    \end{equation}
\end{Property}
Observons que:
\begin{itemize}
    \item Puisque le potentiel est une fonction paire, les fonctions propres de H possèdent une parité définie. On peut alors rechercher les fonctions propres de H parmi les fonctions ayant une parité définie. 
    \item Le spectre d'énergie est discret. 
\end{itemize}
Nous allons à présent tenter de retrouver ces résultats.
\subsection{Valeurs propres de l'Hamiltonien}
Nous allons tenter de résoudre l'équation aux valeurs propres 
\begin{equation}
    H\ket{\psi} = E\ket{\psi},
\end{equation}
c'est à dire tenter de déterminer le spectre et les valeurs propres de l'Hamiltonien.\\

Si $\hat{X}$ et $\hat{P}$ étaient des nombres et non des observables, nous pourrions réécrire leur somme quadratique dans $\eqref{Hamiltonien quantique hat}$ sous la forme $(\hat{X}-i\hat{P})(\hat{X}+i\hat{P})$ : comme ce sont des opérateurs, ils ne commutent en générale pas\footnote{C'est bien le cas ici ; voir la valeur du commutateur de X et de P.}. Nous allons montrer que l'introduction d'opérateurs proportionnels à $\hat{X}$ et à $\hat{P}$ permet de simplifier la recherche des vecteurs et valeurs propres de $\hat{H}$. On pose alors
\begin{align}
    \hat{a} &= \frac{1}{\sqrt{2}}(\hat{X}+i\hat{P})   & \hat{X} &= \frac{1}{\sqrt{2}}(\hat{a}+\hat{a}^\dagger)\label{a et X hat}\\
    \hat{a}^\dagger &= \frac{1}{\sqrt{2}}(\hat{X}-i\hat{P})    & \hat{P} &= \frac{i}{\sqrt{2}}(\hat{a}^\dagger-\hat{a}) \label[]{a dag et P hat}
\end{align}

Il s'agit des opérateurs d'échelle, respectivement opérateurs d'anhilihation et de création. Observons que $[\hat{a},\hat{a}^\dagger] = \mathcal{I}$, $[\hat{a},\hat{a}] = 0 = [\hat{a}^\dagger,\hat{a}^\dagger]$. En introduisant le nombre $N = a^\dagger a = \frac{1}{2}\left(X^2+P^2-1\right)$, nous avons donc
\begin{equation}
    \hat{H} = N+\frac{1}{2}.
\end{equation}
Les vecteurs propres de $\hat{H}$ sont les vecteurs propres de $N$, et inversement. Avant de passer à la détermination du spectre, effectuons quelques observations:
\begin{Property}
    N est hermitien : $N^\dagger = a^\dagger \left(a^\dagger\right)^\dagger$.
\end{Property}
\begin{Property}
    \label{N positif}
    Les valeurs propres de N sont positives.
\end{Property}
\begin{proof}
    Soit $\ket{\varphi}$ une valeur propre de $N$. Dès lors,
    \begin{align*}
        \braket{\varphi|a^\dagger a|\varphi} &= \norm{a\ket{\varphi}}^2 \geq 0
    \end{align*}
\end{proof}

\subsection{Analyse des valeurs et vecteurs propres de N}

\begin{Property}
    $[N,a]=-a$ et $[N,a^\dagger] = a^\dagger$
\end{Property}
\begin{proof}
    \begin{align*}
        [N,a] &= [a^\dagger a,a] = [a^\dagger,a]a+a^\dagger [a,a] = -a\\
        [N,a^\dagger] &= [a^\dagger,a^\dagger]a+a^\dagger [a,a^\dagger] = a^\dagger
    \end{align*}
    Ce qui prouve l'assertion.
\end{proof}

\begin{Property}
    \label{Vp destruction}
    Soit $\ket{\varphi}$ un vecteur propre de N de valeur propre $\nu$ : $N\ket{\varphi} = \nu\ket{\varphi}$. Alors,
    \begin{itemize}
        \item $a\ket{\varphi}$ est vecteur propre de N de valeur propre $\nu-1$.
        \item Si $\nu = 0$, alors $a\ket{\varphi} = 0$.
    \end{itemize}
\end{Property}
\begin{proof}
    \begin{align*}
        N a\ket{\varphi} &= (aN-a)\ket{\varphi} = (a\nu-a)\ket{\varphi} = (\nu-1)a\ket{\varphi}\\
        \norm{a\ket{\varphi}} &= \braket{\varphi|a^\dagger a|\varphi} = \nu \braket{\varphi|\varphi} = 0 \iff \nu = 0
    \end{align*}
\end{proof}

\begin{remark}
    Cela justifie le nom que porte l'opérateur $\hat{a}$ : l'opérateur destruction.
\end{remark}

\begin{Property}
    Soit $\ket{\varphi}$ un vecteur propre de N de valeur propre $\nu$ : $N\ket{\varphi} = \nu\ket{\varphi}$. Alors,
    \begin{itemize}
        \item $a^\dagger\ket{\varphi}$ est non nul.
        \item $a^\dagger\ket{\varphi}$ est un vecteur propre de valeur propre $\nu+1$.
    \end{itemize}
\end{Property}
\begin{proof}
    \begin{align*}
        N \left(a^\dagger \ket{\varphi}\right) &= \left(a^\dagger N + a^\dagger\right)\ket{\varphi} = \left(\nu+1\right)a^\dagger\ket{\varphi}\\
        \norm{a\ket{\varphi}}^2 &= \braket{\varphi|a^\dagger a|\varphi} = \braket{\varphi|a^\dagger a+1|\varphi}\\
        &= \braket{\varphi|N+1|\varphi} = \left(\nu+1\right)\braket{\varphi|\varphi} \geq 0
    \end{align*}
\end{proof}

\begin{remark}
    Cela justifie le nom que porte l'opérateur $\hat{a}^\dagger$ : l'opérateur création.
\end{remark}

\begin{Property}
    Soit $N=aa^\dagger$. Alors, le spectre de N $\subseteq \mathbb{N}$.
\end{Property}
\begin{proof}
    Supposons par l'absurde que $N\ket{\varphi} = \nu\ket{\varphi}$ et que $n < \nu < n+1$. En vertue de la propriété $\ref{Vp destruction}$, $a^{n+1}\ket{\varphi}$ est non nul et est vecteur propre de N de valeur propre $\nu-n-1 < 0$: cela constitue une contradiction avec la propriété $\ref{N positif}$. La conclusion s'ensuit.
\end{proof}

\begin{corollary}
    Si il existe un vecteur propre $\ket{\varphi}$ de valeur propre $\nu\in\mathbb{N}$, alors le spectre de $N$ est $\mathbb{N}$.
\end{corollary}

\begin{Property}
    \label{Niveaux harmonique non degeneres}
    L'état fondamental d'un oscillateur harmonique est non dégénéré.
\end{Property}

\begin{proof}
    En vertue de $\ref{Vp destruction}$, la valeur propre associée à l'état fondamental est 0, de sorte que tout vecteur propre fondamental doive respecter 
    \begin{equation}
        a\ket{\varphi} = 0
    \end{equation}
    En rappelant les défintions $\eqref{a et X hat}$ et $\eqref{a dag et P hat}$, 
    \begin{align*}
        \frac{1}{\sqrt{2}}\left(\hat{X}+i\hat{P}\right)\ket{\varphi} &= 0\\
        \frac{1}{\sqrt{2}}\left(x\varphi(x)+\h\partial_x \varphi(x)\right) &= 0
    \end{align*}
    L'unique solution de ce système est donné par 
    \begin{equation}
        \varphi(x) = ce^{-\frac{x^2}{\h}}
    \end{equation}
    où c est une constante d'intégration.
\end{proof}

\begin{Property}
    Tous les niveaux d'un oscillateur harmonique sont non dégénérés.
\end{Property}
\begin{proof}
    Supposons par l'absurde qu'il existe un niveau $n\in\mathbb{N}$ tel que $\ket{\varphi},\ket{\varphi'}$ sont des vecteurs propres de N, c'est à dire tel que 
    \begin{align}
        N\ket{\varphi} &= n\ket{\varphi}\\
        N\ket{\varphi'} &= n\ket{\varphi'}\\
        \braket{\varphi|\varphi'} &= 0
    \end{align}
    Il s'ensuit que $a\ket{\varphi}$ et $a\ket{\varphi'}$ sont des vecteurs propres de N de valeur propre $n-1$, orthogonaux.
    \begin{equation*}
        \left(\bra{\varphi'}a^\dagger\right)\left(a\ket{\varphi}\right) = \braket{\varphi'|N|\varphi} = n\braket{\varphi'|\varphi} = 0
    \end{equation*}
    Par récurrence, on voit que $a^n\ket{\varphi}$ et $a^n\ket{\varphi'}$ sont des vecteurs propres de N de valeur propre 0, orthogonaux. Cela constitue une contradiction avec $\ref{Niveaux harmonique non degeneres}$.
\end{proof}

Nous pouvons constuire une base orthonormée de vecteurs propres selon 
\begin{equation}
    \ket{n} = \frac{\left(a^\dagger\right)^n}{\sqrt{n!}}\ket{0}
\end{equation}
Cela est la construction d'une base orthonormée dans l'espace de $\mathcal{H}$ilbert: cette base porte le nom de Base de $\mathcal{F}$ock. Nous avons dès lors la relation de fermeture
\begin{equation}
    \sum_{n = 0}^{\infty} \ket{n}\bra{n} = \mathbb{I}
\end{equation}
De plus, le spectre de l'oscillateur harmonique est donné par 
\begin{equation}
    \h\omega\left(N+\frac{1}{2}\right)
\end{equation}
pour tout $n\in\mathbb{N}$. L'énergie de point zéro, comme nous le montrons en $\ref{Application à l'Oscillateur Harmonique}$, est alors donné par
\begin{equation}
    E_0 = \frac{1}{2}\h\omega
\end{equation}

\subsection{Evolution temporelle}
Effectuons à présent une directe application du $\ref{Postulat 5}$ de la mécanique quantique. Considérons un oscillateur harmonique dont l'état est donné à l'instant $t = 0$ par
\begin{equation}
    \ket{\psi (0)} = \sum_{n = 0}^{\infty} c_n\ket{n}
\end{equation}
En vertue du postulat d'évolution des états, nous avons alors que l'état du système à un temps t sera donné par
\begin{equation}
    \ket{\psi (t)} = \sum_{n = 0}^{\infty} c_n e^{-i \frac{E_n t}{\h}}\ket{n} = \sum_{n = 0}^{\infty} c_n e^{-i\left(n+\frac{1}{2}\right)\omega t}\ket{n}
\end{equation}

La valeur moyenne au cours du temps d'une grandeur physique A est alors donnée par
\begin{equation}
    \braket{\psi(t)|A|\psi(t)} = \sum_m \sum_n c_m^* (0)c_n (0)A_{mn}e^{i\left(m-n\right)\omega t}
\end{equation}

\color{red} \textit{Je ne comprend pas la suite des notes de cette section.}

\color{black}

\subsection{Fonctions d'ondes de l'oscillateur harmonique}

Une démonstration par récurrence exploitant l'opérateur de création $\hat{a}^\dagger$ montre que les états propre de l'opérateur nombre $\hat{N} = \hat{a}^\dagger\hat{a}$ s'écrivent
\begin{equation}
    \ket{n}=\frac{\left(a^\dagger\right)^n}{\sqrt{n!}}\ket{0}
\end{equation}
Dans la représentation position, il suffit de substituer l'expression de $\hat{a}$ et de $\psi_0 (x) = \braket{x|0}$ pour obtenir l'expression de la fonction d'onde $\psi_n (x) = \braket{x|n}$, soit
\begin{equation}
    \psi_n (x) = \frac{1}{\sqrt{2^n n!}}\left(x-\partial_x\right)^n\psi_0(x) = \frac{1}{\pi^{\frac{1}{4}}}\frac{e^{-\frac{x^2}{2}}}{\sqrt{2^nn!}}H_n(x)
\end{equation}
où $H_n(x)$ est le polynôme d'Hermite, définie par $H_n(x) = \left(x-\frac{d}{dx}\right)^n$. \color{red} D'où vient le facteur $\frac{1}{\pi^\frac{1}{4}}$ ? \color{black}

\subsection{Résolution de l'équation aux valeurs propres par la méthode polynômiale}
\subsubsection{Forme asymptotique de $\psi(x)$}

Nous voulons résoudre l'équation de l'oscillateur harmonique analytiquement : nous n'allons pas utiliser les opérateurs de création et de destruction.\\

Dans la représentation position, l'équation aux valeurs propres de H s'écrit 
\begin{equation}
    \label{VP H}
    \frac{1}{2}\left(-\frac{d^2}{dx^2}+x^2\right)\psi = E\psi 
\end{equation}

Observons que le potentiel est paire : dès lors, les solutions sont soit paires soit impaires.\\

Nous pouvons réécrire l'équation $\eqref{VP H}$ sous la forme
\begin{equation}
    \label{Equa etudiee}
    \big\{\frac{d^2}{dx^2}-\left(x^2-2E\right)\big\}\psi = 0
\end{equation}
Recherchons des solutions intuitives de cette équation pour des x très grands. Observons que les fonctions
\begin{equation}
    G_{\pm}(x) = e^{\pm \frac{x^2}{2}}
\end{equation}
sont solutions des équations différentielles
\begin{equation}
    \big\{\frac{d^2}{dx^2}-(x^2\pm 1)\big\}G_{\pm}(x) = 0
    \label{Equa grand x}
\end{equation}
\begin{remark}
    Observons que lorsque x tend vers l'infini,
    \begin{align*}
        x^2\pm 1 \sim x^2 \sim x^2-2E
    \end{align*}
\end{remark}

Dès lors, les solutions des équations $\eqref{Equa etudiee}$ et $\eqref{Equa grand x}$ ont la même forme pour des grands x. On s'attend donc que, sous cette hypothèse, que les solutions de $\eqref{Equa etudiee}$ soient de la forme
\begin{align}
    \label{Solution equa etudiee}
    \varphi_-(x) &= e^{-\frac{x^2}{2}} & &\underbrace{\varphi_+(x) = e^{\frac{x^2}{2}}}_{\text{A exclure !}}
\end{align}

En comparant $\eqref{Solution equa etudiee}$ et $\eqref{Equa etudiee}$, nous obtenons le résultat
\begin{equation}
    \label{solution}
    \frac{d^2}{dx^2}\varphi_-(x) \color{red}- 2x\frac{d}{dx}\varphi_-(x) \color{black} + \left(2E-1\right)\varphi_-(x) = 0
\end{equation}

Nous allons à présent montrer une technique de résolution de cette équation différentielle consistant à développer en série $\psi(x) \doteq \varphi_-(x)$.

\subsubsection{Calcul de $\psi(x)$ sous forme d'un développement en série entière}

Nous avons vu que les solutions de $\eqref{Equa etudiee}$ sont soit paires, soit impaires. $\psi (x)$ étant paire, nous recherchons une solution de la forme
\begin{equation}
    \psi(x) = \sum_{m = 0}^\infty a_{2m}x^{2m+p}
\end{equation}

Observons alors que
\begin{align}
    \psi' &= \sum_{m = 0}^{\infty}\left(2m+p\right)a_{2m}x^{2m+p-1} & \psi'' &= \sum_{m = 0}^{\infty}\left(2m+p\right)\left(2m+p-1\right)a_{2m}x^{2m+p-2}
\end{align}

Pour que $\eqref{solution}$ soit satisfaite en vertue de nos résultats, il faut que le développement en série du premier membre soit nul terme à terme, c'est à dire qui vérifie
\begin{equation}
    p(p-1)a_0x^{p-2} = 0
\end{equation}
comme $a_0 \neq 0$, on a soit $p = 0$, soit $p = 1$.Les autres termes donnent alors la récurrence
\begin{align}
    \left(2m+p\color{red}+2\color{black}\right)\left(2m+p\color{red}+\color{black}1\right)a_{2m\color{red}+2\color{black}}&= \left(4m+2p+1-2E\right)a_{2m} \\
    a_{2m\color{red}+2\color{black}} &= \frac{\left(4m+2p+1-2E\right)}{\left(2m+p\color{red}+2\color{black}\right)\left(2m+p\color{red}+\color{black}1\right)}a_{2m}  \label{recurrence}
\end{align}

Pour de grands m, nous aurons la relation $a_{2m+2} \approx \frac{1}{m}a_{2m}$. Cela correspond au développement en série de $e^{x^2}$:
\begin{align}
    e^{x^2} &= \sum_m \frac{x^{2m}}{m!} = \sum_m c_{2m}x^{2m} & \frac{c_{2m+2}}{c_{2m}} &= \frac{1}{m}
\end{align}

Il s'agit d'un comportement asymptotique $e^{-\frac{x^2}{2}}e^{x^2} = e^{\frac{x^2}{2}}$ : cela n'est pas acceptable physiquement. La série n'a pas de forme asymptotique si et seulement si la récurrence $\eqref{recurrence}$ se termine après un nombre fini de termes. On considère alors une énergie $E_m$ tel que $4m+2p+1-2E_m = 0$ où $p = 0,1$. Dès lors, nous déduisons la quantification de l'énergie de l'oscillateur harmonique quantique:
\begin{align}
    E_n &= n+\frac{1}{2} & n &= 2m+p
\end{align}

\begin{remark}
    Le coefficient $a_0$ n'est pas déterminé par la recurrence : nous le choisissons de sorte à normaliser la solution.
\end{remark}

\end{document}