\documentclass[../notesdecours.tex]{subfiles}

\begin{document}
\part{Oscillateur Harmonique Quantique}
\section{De l'importance de l'Oscillateur Hamonique}
L'importance de l'Oscillateur Harmonique en Physique ne peut pas être sous-estimé. Des exemples d'applications sont légion ; prenont la $\mathcal{M}$écanique $\mathcal{C}$lassique pour l'exemple.\\

Le plus simple reste de considérer une particule de masse $m$ se déplaçant dans un potentiel central de la forme 
\begin{align}
    V(x) &= \frac{1}{2}kx^2     &\forall k\in\mathbb{R^+}
\end{align}
Dès lors, la particule effectue un mouvement oscillatoire autour du plan $x = 0$, avec une force de rappel 
\begin{equation}
    F_x = -\frac{dV}{dx} = -kx
\end{equation}
Cette situation est régie par l'équation d'un Oscillateur Harmonique, soit
\begin{equation}
    m\ddot{x} = -kx
\end{equation}
On pose alors souvent $\omega = \sqrt{\frac{k}{x}}$ ; il s'agit de la pulsation du mouvement. La solution générale de cette équation est donnée par la relation 
\begin{align}
    x(t) &= Acos \left(\omega t-\varphi\right)       &\forall A\in\mathbb{R}^+,\forall\varphi\in [0,2\pi]
\end{align}
En particulier, nous avons que l'énergie totale de la particule s'exprime par la relation
\begin{equation}
    E = T+V = \frac{1}{2}m\omega^2A^2 = \frac{p^2}{2m}+\frac{kx^2}{2} \doteq H
\end{equation}

\begin{remark}
    L'oscillateur harmonique joue un rôle fondamental en Physique ; il permet de décrire (du moins, de manière rapprochée) les mouvements d'osccilations autour d'une position d'équilibre.
\end{remark}

\begin{remark}
    En Mécanique Quantique, l'Oscillateur Harmonique est le problème exactement soluble ayant le plus d'applications.
\end{remark}

\section{L'Oscillateur Harmonique en Mécanique Quantique}
Dans les discussions quantiques, nous remplacons les grandeurs classiques $x$ et $p$ par les observables $X$ et $P$, vérifiant la relation $[X,P]=i\h$ (voir le chapitre $\ref{Chapitre 6}$ pour plus de détails et une preuve détaillée). L'Hamiltonien quantique est donc donné par 
\begin{equation*}
    H = \frac{P^2}{2m}+\frac{kX^2}{2}
\end{equation*}

Effectuons - pour facilier les notations - les transformations canoniques suivantes:
\begin{subequations}
    \begin{equation}
        \hat{X} = \sqrt{\frac{m\omega}{\h}}X
    \end{equation}
    \begin{equation}
        \hat{Y} = \frac{1}{\sqrt{m\h\omega}}P
    \end{equation}
\end{subequations}
En particulier, nous avons alors que $[\hat{X},\hat{X}]=i$, et que $H=\h\omega\hat{H}$, où
\begin{equation}
    \hat{H} = \frac{1}{2}\left(\hat{X}^2+\hat{P}^2\right).
    \label{Hamiltonien quantique hat}
\end{equation}
Observons que:
\begin{itemize}
    \item Puisque le potentiel est une fonction paire, les fonctions propres de H possèdent une parité définie. On peut alors rechercher les fonctions propres de H parmi les fonctions ayant une parité définie. 
    \item Le spectre d'énergie est discret. 
\end{itemize}
Nous allons à présent tenter de retrouver ces résultats.
\subsection{Valeurs propres de l'Hamiltonien}
Nous allons tenter de résoudre l'équation aux valeurs propres 
\begin{equation}
    H\ket{\psi} = E\ket{\psi},
\end{equation}
c'est à dire tenter de déterminer le spectre et les valeurs propres de l'Hamiltonien.\\

Si $\hat{X}$ et $\hat{P}$ étaient des nombres et non des observables, nous pourrions réécrire leur somme quadratique dans $\eqref{Hamiltonien quantique hat}$ sous la forme $(\hat{X}-i\hat{P})(\hat{X}+i\hat{P})$ : comme ce sont des opérateurs, ils ne commutent en générale pas\footnote{C'est bien le cas ici ; voir la valeur du commutateur de X et de P.}. Nous allons montrer que l'introduction d'opérateurs proportionnels à $\hat{X}$ et à $\hat{P}$ permet de simplifier la recherche des vecteurs et valeurs propres de $\hat{H}$. On pose alors
\begin{align}
    a &= \frac{1}{\sqrt{2}}(X+iP)   & X &= \frac{1}{\sqrt{2}}(a+a^\dagger)\\
    a^\dagger &= \frac{1}{\sqrt{2}}(X-iP)    & P &= \frac{i}{\sqrt{2}}(a^\dagger-a)
\end{align}
Observons que $[a,a^\dagger] = 1$. En introduisant le nombre $N = a^\dagger a = \frac{1}{2}\left(X^2+P^2-1\right)$, nous avons donc
\begin{equation}
    \hat{H} = N+\frac{1}{2}.
\end{equation}
Les vecteurs propres de $\hat{H}$ sont les vecteurs propres de $N$, et inversement. Avant de passer à la détermination du spectre, effectuons quelques observations:
\begin{itemize}
    \item N est hermitien : $N^\dagger = a^\dagger \left(a^\dagger\right)^\dagger = a^\dagger a = N$.
    \item N est positif : $\forall \ket{\varphi}$,
    \begin{align*}
        \braket{\varphi|N|\varphi} &= \braket{\varphi|a^\dagger a |\varphi}\\
        &= \norm{a\ket{\varphi}}^2\\
        &\geq 0
    \end{align*}
    Les valeurs propres de N sont également positives.
\end{itemize}

\subsection{Analyse des valeurs et vecteurs propres de N}

\begin{Property}[Coucou]
    $[N,a]=-a$ et $[N,a^\dagger] = a^\dagger$
\end{Property}
\begin{proof}
    \begin{align*}
        [N,a] &= [a^\dagger a,a] = [a^\dagger,a]a+a^\dagger [a,a] = -a\\
        [N,a^\dagger] &= [a^\dagger,a^\dagger]a+a^\dagger [a,a^\dagger] = a^\dagger
    \end{align*}
    Ce qui prouve l'assertion.
\end{proof}

\begin{Property}
    Soit $\ket{\varphi}$ un vecteur propre de N de valeur propre $\nu$ : $N\ket{\varphi} = \nu\ket{\varphi}$. Alors,
    \begin{itemize}
        \item $a\ket{\varphi}$ est vecteur propre de N de valeur propre $\nu-1$.
        \item Si $\nu = 0$, alors $a\ket{\varphi} = 0$.
    \end{itemize}
\end{Property}
\begin{proof}
    \begin{align*}
        N a\ket{\varphi} &= (aN-a)\ket{\varphi} = (a\nu-a)\ket{\varphi} = (\nu-1)a\ket{\varphi}\\
        \norm{a\ket{\varphi}} &= \braket{\varphi|a^\dagger a|\varphi} = \nu \braket{\varphi|\varphi} = 0 \iff \nu = 0
    \end{align*}
\end{proof}
\begin{Property}
    Soit $\ket{\varphi}$ un vecteur propre de N de valeur propre $\nu$ : $N\ket{\varphi} = \nu\ket{\varphi}$. Alors,
    \begin{itemize}
        \item $a^\dagger\ket{\varphi}$ est non nul.
        \item $a^\dagger\ket{\varphi}$ est un vecteur propre de valeur propre $\nu+1$.
    \end{itemize}
\end{Property}
\begin{proof}
    \begin{align*}
        N \left(a^\dagger \ket{\varphi}\right) &= \left(a^\dagger N + a^\dagger\right)\ket{\varphi} = \left(\nu+1\right)a^\dagger\ket{\varphi}\\
        \norm{a\ket{\varphi}}^2 &= \braket{\varphi|a^\dagger a|\varphi}
    \end{align*}
\end{proof}
\end{document}