\documentclass[../notesdecours.tex]{subfiles}

\begin{document}
\appendix
\part{Appendice}
\section{Résultats élémentaires d'Algèbre Linéaire}
\label{AppendeiceA}
Rappelons une série de résultats classiques d'Algèbre Linéaire pertinents à la Mécanique Quantique.\\

\begin{definition}[Produit Hermitien] Soit V un espace vectoriel sur $\mathbb{C}$. On y définit le produit hermitien, c'est à dire une application
\begin{center}
\begin{tabular}{c c c}
$V \times V$ & $\rightarrow$ & $\mathbb{C}$\\
$\left(\bm{x},\bm{y}\right)$ &  $\rightarrow$ & $\bm{x} \cdot \bm{y}$
\end{tabular}
\end{center}
tel que $\forall$ x,y,x',y' $\in$ V, et tout $\lambda \in \mathbb{C}$,
\begin{enumerate}
\item $y\cdot x = \bar{x} \cdot \bar{y}$
\item $\left( x+x' \right) \cdot y = x \cdot y + x' \cdot y$, et $x\cdot(y+y') = x\cdot y+x\cdot y'$
\item $\left( \lambda x \right)\cdot y = \lambda \left(x\cdot y\right)$ et $x\cdot (\lambda y) = \bar{\lambda} (x\cdot y)$
\item $x \cdot x \in \mathbb{R}_{\geq 0} \forall x, et x \cdot x = 0$ si et seulement si $x = 0$.
\end{enumerate}
Un espace hermitien est un espace vectoriel V sur $\mathbb{C}$ muni d'un produit hermitien.
\end{definition}

\begin{Property}
Soit V un espace Hermitien de dimension n. Si E $\doteq \left(e_1,...,e_n\right)$ est un ensemble de vecteurs deux à deux orthogonales, alors E est une base de V. 
\end{Property}


\begin{Property} Soit V un espace Hermitien. Alors il existe une base orthonormale V. \end{Property}

Nous pouvons utiliser l'algorithme de Gram-Schmidt pour ortogonaliser une base de V d'un espace vectoriel sur $\mathbb{C}$ ou $\mathbb{R}$.\\

\begin{definition} Une matrice $a \in$ GL($V_{\mathbb{C}}$) est unitaire si $a^{-1} = \bar{a}^T$. L'ensemble des matrices unitaires de taille $n \times n$ est dénotée par $U_n$. \label{Unitaire}\end{definition}

\begin{definition} Une matrice $a \in$ Mat($\mathbb{C}$) est Hermitienne si $\bar{a}^T = a$. \label{Hermitienne}\end{definition}

\begin{Property} A est une isométrie si et seulement si a est unitaire (si $V_\mathbb{C}$). \end{Property}

Voici une série de propriétés classiques des isométries:
\begin{enumerate}
\item Les isométries conservent les distances (normes) et les angles.
\item Supposons que E est orthonormale. Alors A est une isométrie si et suelement si les vecteur qui forment les colonnes de a sont:
	\begin{enumerate}
	\item deux à deux orthogonaux
	\item de norme 1.
	\end{enumerate}
\item Si $\lambda$ est une valeur propre de A, alors $\norm{\lambda} = 1$.
\item Si A est une isométrie, alors $\norm{det(a)} = 1$.
\item Si E et F sont des bases orthonormales de V, alors il existe une unique isométrie A tel que $A(e_i) = f_i$.
\item Tous les éléments de $O_3$ sont d'un des trois types suivants:
	\begin{enumerate}
	\item Rotations autour d'une droite passant par l'origine.
	\item Symétries par rapport à un plan passant par l'origine.
	\item Une composition d'isométries de type (I) et (II).
	\end{enumerate}
\end{enumerate}

\begin{lemma} Toutes les valeurs propres d'une matrice Hermitienne sont réelles. \end{lemma}


\begin{theorem} Soit $a \in Mat_{n \times n} (\mathbb{C})$ Hermitienne. Il existe une base orthonormale de V contenant que des vecteurs propres de a. En d'autres mots, il existe une matrice O, unitaire, tel que

\begin{equation}
O^{-1}aO = \bar{O}^TaO
\end{equation}
\end{theorem}

\begin{definition}
Soit $\mathbb{H}$ un espace de Hilbert. $\mathbb{H}$ est séparable si il possède une base dénombrable. 
\end{definition}

\begin{remark}
Soit $u_i$ une base $\forall i\in\mathbb{N}$. Par Gram-Schmidt, nous pouvons prendre la base orthonormée $(u_i,u_j) = \delta_{ij}$.
\end{remark}


\section{Approximation BKW}

\color{blue} \textbf{En cours de réaction.}

% Soit $\psi$ la fonction d'onde solution de l'équation de Schrodinger d'une particule de masse $m$ dans un potentiel central $V(\bm{r})$. L'approximation BKW consiste à écrire la fonction d'onde sous la forme
% \begin{equation}
% 	\psi (\bm{r}) = \frac{C_+}{\sqrt{\norm{\psi(\bm{r})}}}e^{\frac{i}{\h}\int p} + \frac{C_-}{\sqrt{\norm{\psi(\bm{r})}}}e^{-\frac{i}{\h}\int p}
% \end{equation}
% où $\psi(\bm{r}) = \sqrt{2m\left(E-V(\bm{r})\right)}$ est l'impulsion locale de la particule.

% \subsection{Application à l'équation de Schrodinger indépendant du temps à une dimension}

% Nous voulons résoudre l'équation 
% \begin{equation*}
% 	\frac{d^2 \psi(x)}{dx^2} = \frac{2m}{\h^2}\left(V(x)-E\right)\psi(x)
% \end{equation*}
% Prenons l'ansatz standard $\Psi(x) = e^{\Phi(x)}$ (avec $\Phi(x)$ un coefficient potentiellement complexe), de sorte à avoir \color{red}par quelle magie? \color{black}
% \begin{equation}
% 	\Phi''(x) + \left(\Phi'(x)\right)^2 = \frac{2m}{\h^2}(V(x)-E)
% \end{equation}

% Nous pouvons écrire $\Phi'(x)$ en ses parties réelles et complexes en introduisant les fonctions réelles $A(x)$ et $B(x)$,
% \begin{equation}
% 	\Phi'(x) = A(x) + iB(x)
% \end{equation}

% L'amplitude de la fonction d'onde est dès lors exprimée par la reation $\exp \{\int_{x_0}^{x}A(x') \; dx'\}$ et la phase par la relation (similaire) $\int_{x_0}^{x} B(x') \; dx'$. L'équation de Schrodinger se réécrit alors, assez simplement, en
% \begin{subequations}
% 	\begin{equation}
% 		A'(x) + A^2 (x)-B^2 (x) = \frac{2m}{\h^2}\left(V(x)-E\right),
% 	\end{equation}
% 	\begin{equation}
% 		B'(x)+2A(x)B(x) = 0.
% 	\end{equation}
% \end{subequations}
% Au $0^{eme}$ degré, nous devons imposer les conditions
% \begin{align}
% 	A_0(x)^2 - B_0(x)^2 &= \frac{2m}{\h}\left(V(x)-E\right) & A_0(x)B_0(x) &= 0 
% \end{align}

% En supposant que la phase varie plus rapidement que l'amplitude (c'est à dire que $A_0(x) = 0$), nous aurons que
% \begin{equation}
% 	B_0(x) = \pm \sqrt{2m\left(E-V(x)\right)}
% \end{equation}

% Dans ce cas, nous aurons 
\end{document}