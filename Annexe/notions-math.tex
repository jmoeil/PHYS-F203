\documentclass[../notesdecours.tex]{subfiles}

\begin{document}
\chapter{Notions mathématiques}
\section{Série de Fourier}
Une série de Fourier est une série de la forme
\begin{subequations}
\begin{equation}
c_n = \frac{1}{T} \int^{T/2}_{-T/2} dx f(x)e^{-2\pi i(\frac{n}{T})x}
\end{equation}
\begin{equation}
f(x) = \sum_{n = -\infty}^{+\infty}c_ne^{2\pi i (\frac{n}{T})x}
\end{equation}
\end{subequations}

\section{Transformées de Fourier}
\begin{subequations}
\begin{equation}
\hat{f}(k) = F(k) = \int_{-\infty}^{+\infty} dx f(x) \frac{e^{-i\bm{k}}}{\sqrt{2\pi}}
\end{equation}
\begin{equation}
f(x) = F^{-1}(\hat{h}) = \int d\bm{k}\hat{f}(\bm{k} \frac{e^{i\bm{k}x}}{\sqrt{2\pi}}
\end{equation}
\end{subequations}
\begin{remark}
Si $f$ est à support borné et $\{-\frac{T}{2},\frac{T}{2}\}$ contraint le support, alors $C_n = \frac{\sqrt{2\pi}}{T}\hat{f}(\frac{2\pi n}{T})$.
\end{remark}
\begin{align}
\rightarrow f(x) &= \sum_{n = -\infty}^{+\infty} \frac{\sqrt{2\pi}}{T}\hat{f}(\frac{2\pi n}{T})e^{i\frac{2\pi n}{T}x}\\
&= \sum_{n = -\infty}^{+\infty} \hat{f}(k_n)\frac{e^{ik_nx}}{\sqrt{2\pi}} \Delta k		&k_n = \frac{2\pi n}{T} \textbf{ et } \Delta k = \frac{2\pi}{T}\\
&\approx \int_{-\infty}^{+\infty} dk \hat{f}(k) \frac{e^{ikx}}{\sqrt{2\pi}}
\end{align}
\begin{remark}
Ces fonctions suivent certaines propriétés intéressantes. Soit $h(x)$ et $\hat{h}(x)$ deux fonctions reliées par une transformations de Fourier. Dès lors,
\begin{itemize}
\item Si $h(x)$ est linéaire, alors $\hat{h}(x)$ l'est également : $h(x) = af(x) + bg(x)$, alors $\hat{h}(k) = a\hat{f}(k) + b\hat{g}(k)$.
\item Si $h(x) = f(x-x_0)$, alors $\hat{h}(k) = e^{-ikx_0}\hat{f}(k)$. Il s'agit d'une translation. Inversement, la propriété de modulation s'écrit $h(x) = f(x)e^{ik_0x}$, alors $\hat{h}(k) = \hat{f}(k-k_0)$.
\item Si $h(x) = f(ax)$, le changement d'échelle implique que $\hat{h}(k) = \frac{1}{a}\hat{f}(\frac{k}{a})$.
\item La relation de conjuguaison sous une transformation de Fourier est que $h(x) = \bar{f}(x)$ implique $\hat{h}(k) = \bar{\hat{f(-k)}}$. Notons que si $f(x)$ est réel, alors $\hat{f}(k) = -\hat{f}(k)$.
\item $\hat{f}(0) = \int_{-\infty}^{\infty} dxf(x)$.
\item La dérivée de $\hat{f}(k)$ est $ik\hat{f}(k)$. Cela se généralise à $\hat{f^(n)} = (ik)^n\hat{f}(k)$. En particulier, si $f(x)x^n$ est intégrable, alors $\hat{f}(k)$ est n-fois dérivable. Inversement, si $f(x)$ est n-fois intégrable, alors $\hat{f}(k)k^n$ est intégrable.
\item La propriété de convolution établit que si $h(x) = (f\circ h)(x) = \int dy f(y)g(x-y)$, alors $\hat{h}(k) = \hat{f}(k)\ti\hat{g}(k)$.
\end{itemize}

\end{remark}
\begin{theorem}[Plancherel]
Soit $f(x)$ une fonction, et $\hat{f}(k)$ sa transformée de Fourier. Nous avons alors l'équivalence des intégrales:
\begin{equation}
\int dx f(x)\bar{g}(x) = \int dk \hat{f}(k)\bar{\hat{g}}(k)
\end{equation}
\end{theorem}
\begin{theorem}[Égalité de Parceval]
Soit $f(x)$ une fonction, et $\hat{f}(k)$ sa transformée de Fourier. Alors,
\begin{equation}
\int dx \norm{f(x)}^2 = \int dk \norm{\hat{f}(k)}^2
\end{equation}
\end{theorem}

\section{Distribution}
\subsection{Espace de fonctions test}
Soient $D$, l'ensemble des fonctions $C^\infty$ à support compact (distrubution D'), et S - l'ensemble des fonctions $C^\infty$ à décroissance rapide (distrubtion tempérée S').  Imposons une notion de continuité/topologie sur les fonctions test:
\begin{equation}
\varphi_k = \varphi \text{ si et seulement si } (\partial_x^{(\alpha)} \varphi_x) = (\partial_x^{(\alpha)}\varphi)
\end{equation}
uniformément pour tout $\alpha$.\\

Soit T des formes linéaires continues sur l'espace des fonctions tests.
\begin{Property}
Soit $T: D \rightarrow \mathbb{R}$ : $\varphi \rightarrow T\cdot\varphi$. Si $\varphi_k = \varphi$, alors $T\cdot\varphi_k \rightarrow T\cdot\varphi$ généralise la notion de fonction.
\end{Property}

\subsection{Opérations sur les distributions}
\begin{Property}[Dérivée d'une distrubution]
$T'\cdot\varphi = T\cdot(-\varphi')$
\end{Property}
\begin{Property}[Multiplication d'une distribution par une fonction test]
$\Phi T\cdot\varphi = T\cdot\varphi\Phi$
\textbf{Nous ne pouvons pas multiplier des distributions entre-elles.}
\end{Property}
\begin{theorem}[Théorème de structure]
Localement, une distrubution est égale à la dérivée $\alpha^{eme}$ d'une fonction continue. Elle est dite tempérée lorsqu'elle est égale à la dérivée $\alpha^{eme}$ d'une fonction continue à croissance lente\footnote{ne croissant pas plus vite qu'un polynome.}.
\end{theorem}
\subsection{Distributions tempérées}
A partir de maintenant, nous noterons F une transformée de Fourier, et $\mathbb{S}$ une invariance sous F.
\begin{definition} Soit $T \in \mathbb{S}$. Alors, FT existe et est défini par $FT\cdot\Phi = T\cdot F\Phi$.
\end{definition}

Si f est une fonction, alors:
\begin{align}
FT_f\cdot\Phi &= T_f\cdot F\Phi		&\text{Où }\int dx (\int dx \frac{e^{-ikx}}{\sqrt{2\pi}}f(x))\Phi(k) \text{ et } \int dx f(x) (\int dk \frac{e^{-ikx}}{\sqrt{2\pi}}\Phi(k))
\end{align}

\subsection{Delta de Dirac}
\begin{align}
\delta(x) &= \begin{cases}
+\infty \mbox{ en x = 0}\\
0 \mbox{ en } x\neq 0
\end{cases}		&\int^{+\infty}_{-\infty} dx \delta (x) = 1\\
\delta(x) &= \lim_{x\to 0} f_{\alpha}(x)	&\int^{+\infty}_{-\infty} dx f_\alpha (x) = 1
\end{align}
Où $f_\alpha (x)$ est strictement positif.
\begin{align}
\int_{-\infty}^{+\infty} dx f(x)\delta(x) &= f(0)\\
\int_{-\infty}^{+\infty} dx \delta(\Gamma - x)\delta(x- \zeta) &= \delta (\Gamma - \zeta)\\
\delta '(x) : \int^{+\infty}_{-\infty} \delta '(x)f(x) &= [\delta(x)f(x)]_{-\infty}^{+\infty} - \int_{-\infty}^{+\infty} dx \delta(x)f'(x)\\
&= -f'(0)\\
\int_{-\infty}^{+\infty} dx' \delta(x') &= \theta(x)		&\int dx f(x)\delta(x-a) = f(a)\\
\delta(\alpha x) &= \frac{1}{\norm{\alpha}}\delta(x)	&\delta(g(x)) = \frac{1}{\norm{g'(x_0)}}\delta(x-x_0)
\delta(-x) &= \delta(x)\\
\end{align}

\subsection{Transformée de Fourier d'une fonction périodique}
Si $x(t)$ est une fonction de période T tel que $x(t+T) = x(t)$. Alors $x(t)$ peut-être représenté comme une série de Fourier.
\begin{equation}
x(t) = \sum_{k = -\infty}^{+\infty} c_ke^{2\pi ik\frac{i}{T}}
\label{Z}
\end{equation}
Prenons la transformée de Fourier de \eqref{Z}.
\begin{align}
\hat{x} (\omega ) = \int dt \frac{e^{-i\omega t}}{\sqrt{2\pi}} x(t) &= \int_{k = -\infty}^{+\infty} c_k \int dt \frac{e^{-i\omega t}}{\sqrt{2\pi}}e^{2\pi i k \frac{t}{T}}\\
&= \sum_{k = -\infty}^{+\infty} \frac{c_k}{2\pi} \delta (\omega - \frac{2\pi k}{T})
\end{align}
Nous appelons $\hat{x}(\omega)$ est la somme des deltas espacés de $\frac{2\pi}{T}$.
\end{document}