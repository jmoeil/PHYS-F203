\documentclass[../notesdecours.tex]{subfiles}

\begin{document}
\chapter{Produit tensoriel et intrication quantique}

Nous avons étudié en $\ref{Espace mathématique des fonctions d'onde}$ l'espace mathématique des fonctions d'ondes, en raisonnant sur des fonctions d'ondes. Dans cette discussion, nous avons traité de l'espace mathématique des fonctions d'onde à une et à trois dimensions. Dans ce chapitre, nous nous intéressons aux liens entre ces deux espaces ; il apparaît effectivement que l'espace des fonctions d'ondes $\mathcal{\varepsilon}_{\bm{r}}$ à trois dimensions est une sorte de généralisation de l'espace des fonctions unidimensionnel $\mathcal{\varepsilon}_x$. Existe-t-il une relation entre ces deux espaces?\\

\begin{remark} Soient deux espaces $\varepsilon_1$ et $\varepsilon_2$ de dimension respective $N_1$ et $N_2$. \end{remark}



\end{document}