\documentclass{article}
\usepackage[allcolors=true]{hyperref}
\usepackage{graphicx}
\usepackage{amsmath,amsfonts,amsthm,amssymb}
\usepackage{caption}
\usepackage{multirow}
\usepackage{physics}
\usepackage{tikz}
\usepackage{cleveref}
\usepackage{pgfplotstable}
\usepackage{siunitx}
\usepackage{wrapfig}
\usepackage{graphicx}
\usepackage{subfiles}
\usepackage{bm}
\usepackage{xcolor}
\usepackage[french]{babel}
\usepackage{titlesec}
\usepackage{lmodern}
\usepackage{braket}
\usepackage{chngcntr}
\usepackage{geometry}
\usepackage{enumitem}
\usepackage{calrsfs}

\geometry{
     a4paper,
    total={170mm,257mm},
    left=20mm,
    top=20mm,
 }
 \title{Représentations de la position et de l'impulsion en Mécanique Quantique}
 \date{}
 \begin{document}
 
 \maketitle

 Considérons une particule dans l'espace de Hilbert $L^2(\mathbb{R})$ ou $L^2(\mathbb{R}^3)$, \textit{i.e} l'espace des fonctions de carré sommables. \\

Nous avons vu lors du développement du formalisme de Dirac que l'onde plane $v_p(x) = \frac{1}{\sqrt{2 \pi \hbar}} \; e^{ipx/\hbar}$ constituait une base continue dans laquelle il est possible d'écrire une fonction d'onde de l'espace $L^2(\mathbb{R})$, bien que la fonction de la base même ne soit pas de carré sommable. Dans ce cas-ci, cela revient en effet à décomposer une fonction en mode de Fourier puisqu'alors $$ \psi(x) = \int_{-\infty}^{+\infty} \; dp \; \hat{\psi_p} v_p $$ où $\hat{\psi_p} = (v_p, \psi) = \int_{-\infty}^{+\infty} \; dx \; v_p^* \psi(x)$. \\
Rappelons également que l'on ne doit pas perdre de vue qu'à tout état \textbf{physique} doit être associé une fonction de carré sommable. Ainsi, l'onde plane $v_p$ ne peut pas représenter l'état d'une particule par exemple, ce n'est qu'un intermédiaire de calcul. \\

Dans la suite, nous utiliserons les deux bases continues suivantes : 
\begin{enumerate}
    \item $\left\{ \xi_{x_0}(x) \right\}$ où $\xi_{x_0}(x) = \delta(x-x_0)$ 
    \item $\left\{ v_{p_0}(x) \right\}$ où $v_{p_0}(x) = \frac{1}{\sqrt{2\pi \hbar}} \; e^{ip_0 x/\hbar}$ \\
\end{enumerate}
Nous chercherons à représenter les opérateurs position $\hat{X}$ et impulsion $\hat{P}$ grâce à cette base. En effet, ces deux opérateurs n'ont pas de vecteurs propres dans l'espace de Hilbert, mais nous ferons comme si (cela sera ultérieurement justifié), où l'on notera : 
\begin{itemize}[label = \textbullet]
    \item $\ket{x_0} \equiv$ état propre de l'opérateur $\hat{X}$, correspondant à la "fonction propre" $\delta(x-x_0)$, de valeur propre $x_0$. 
    \item $\ket{p_0} \equiv$ état propre de l'opérateur $\hat{P}$, correspondant à la "fonction propre" $\frac{e^{ip_0 x/\hbar}}{\sqrt{2 \pi \hbar}}$, de valeur propre $p_0$. 
\end{itemize}

\subsection*{Relations d'orthonormalisation et de fermeture}

Par définition du produit scalaire, nous avons que 
\begin{itemize}[label=\textbullet]
    \item $\braket{x_0|x_0'} = \int dx \; \delta (x-x_0) \delta (x-x_0') = \delta (x_0 - x_0')$
    \item $\braket{p_0|p_0'} = \int dx \; \frac{e^{-ip_0 x/\hbar}}{\sqrt{2\pi\hbar}} \frac{e^{ip_0'x/\hbar}}{\sqrt{2\pi\hbar}} = \int dx \frac{e^{-i (p_0 - p_0')x/\hbar}}{2\pi \hbar}	= \int du \; \frac{e^{-i(p_0 - p_0')}}{2\pi} = \delta (p_0-p_0')$ (avec le changement de variable $u = x/\hbar$) \\
\end{itemize}

Les bases que l'on a défini sont donc bien orthonormées au sens large ; ceci nous donne ainsi une \textit{relation d'orthonormalisation}. \\

Puisque $\left\{ \ket{x_0} \right\}$ et $\left\{ \ket{p_0} \right\}$ forment une base orthonormée dans l'espace des états, nous pouvons également écrire les \textit{relations de fermeture} (ou de complétude) suivantes : 
\begin{itemize}[label=\textbullet]
    \item $\int dx_0 \; \ket{x_0}\bra{x_0} = \mathbb{I}$ 
    \item $\int dp_0 \; \ket{p_0}\bra{p_0} = \mathbb{I}$ 
\end{itemize}

\subsection*{Composantes d'un ket}

Considérons un état quantique $\ket{\psi}$, correspondant à la fonction d'onde $\psi(x)$. En exploitant les relations de fermetures définies ci-dessus, nous pouvons alors écrire l'état quantique sous les deux formes suivantes:
\begin{align}
\ket{\psi} &= \int dx_0 \; \ket{x_0} \braket{x_0|\psi} \\
\ket{\psi} &= \int dp_0 \; \ket{p_0} \braket{p_0|\psi}
\end{align}

De plus, nous observons que : 
\begin{align}
    \braket{x_0| \psi} &= \int dx \; \xi_{x_0}^*(x) \psi(x) \; \mbox{ (par définition du produit scalaire)} \\
    &= \int dx \; \delta(x-x_0)\psi(x) \\
    &= \psi(x_0)\\
    \mbox{et } \; \braket{p_0|\psi} &= \bra{p_0}\mathbb{I}\ket{\psi} = \bra{p_0} \left( \int dx \; \ket{x_0} \bra{x_0} \right) \ket{\psi} \\
    &= \int dx \; \braket{p_0|x_0} \braket{x_0|\psi} \; \mbox{ où $\braket{p_0|x_0} = \int dx \; p_0^*(x) \delta(x-x_0) = p_0^*(x_0)$} \\
    &= \int dx \; \frac{e^{-ip_0x_0/\hbar}}{\sqrt{e\pi \hbar}} \psi(x_0) \\
    &= \tilde{\psi}(p_0) \; \mbox{ (transformée de Fourier de $\psi(x)$)} \\
\end{align}

En résumé, nous avons donc obtenu ces deux relations importantes : \textcolor{red}{Ici ce serait bien d'écrire ces deux relations dans une boîte}
\begin{itemize}[label=\textbullet]
    \item $\braket{x|\psi} = \psi(x)$
    \item $\braket{p|\psi} = \tilde{\psi}(p)$
\end{itemize}


\subsection*{Produit scalaire de deux vecteurs}
A l'aide des relations de fermeture, nous allons écrire le produit de deux vecteurs de l'espace de Hilbert $L^2(\mathbb{R})$ ($L^2(\mathbb{R}^3)$). \\
En effet, nous avions déjà défini le produit scalaire dans cet espace comme $(f,g) = \int dx \; f^* g$ ; nous verrons que c'est ce que l'on retrouve bien autant dans la base position que la base impulsion : 

\begin{align}
        \braket{\varphi | \psi} &= \bra{\varphi} \left( \int dx \; \ket{x}\bra{x} \right) \ket{\psi} \\
        &= \int dx \; \braket{\varphi |x}\braket{x|\psi} \\
        &= \int dx \; \varphi^*(x)\psi(x)
\end{align}

\begin{align}
        \braket{\varphi | \psi} &= \bra{\varphi} \left( \int dp \; \ket{p}\bra{p} \right) \ket{\psi} \\
        &= \int dp \; \braket{\varphi |p}\braket{p|\psi} \\
        &= \int dp \; \tilde{\varphi}^*(p)\tilde{\psi}(p)
\end{align}

\subsection*{Opérateurs $\hat{X}$ et $\hat{P}$}

Supposons que l'on ait : \begin{itemize}[label=\textbullet]
    \item $\ket{x} \equiv$ "vecteur propre" de l'opérateur $\hat{X}$ de "valeur propre" $x$ ; $\hat{X}\ket{x} = x\ket{x}$
    \item $\ket{p} \equiv$ "vecteur propre" de l'opérateur $\hat{P}$ de "valeur propre" $p$ ; $\hat{P}\ket{p} = p\ket{p}$
\end{itemize}

Ainsi, nous en déduisons directement que : 
\begin{align}
    \bra{x}\hat{X}\ket{\psi} &= x \braket{x|\psi} = x \; \psi(x) \\
    \bra{p}\hat{P}\ket{\psi} &= p \braket{p|\psi} = p \; \tilde{\psi}(p) \\
\end{align}

Nous allons maintenant calculer 2 quantités ($\bra{x}\hat{P}\ket{\psi}$ et $\bra{\varphi} \hat{P} \ket{\psi}$) qui nous permettront de montrer la relation de commutation canonique $[ \hat{X}, \hat{P}] = i\hbar \; \mathbb{I}$. Cette relation est en effet importante car elle nous permet de retrouver à la limite classique où $\hbar \rightarrow 0$ la relation canonique classique donnée par le crochet de Poisson $\{ X^i,P_j \} = \delta^i_j$ ; ceci deviendra important pour quantifier des systèmes (cours de BA3). \\
Commençons par calculer $\bra{x}\hat{P}\ket{\psi}$ : \textcolor{red}{A vérifier cette partie/ajouter commentaires}

\begin{align}
    \bra{x}\hat{P}\ket{\psi} &= \bra{x} \left( \int dp \; \ket{p} \bra{p} \right) \hat{P} \ket{\psi} \notag \\
    &= \int dp \; \braket{x|p} \bra{p}\hat{P}\ket{\psi} \notag \\
    &= \int dp \; \frac{e^{ipx/\hbar}}{\sqrt{2\pi \hbar}} p \tilde{\psi}(p) \notag \\
    &= -i \hbar \; \partial_x \left( dp \; \frac{e^{ipx/\hbar}}{\sqrt{2 \pi \hbar}} \tilde{\psi}(p) \right) \notag \\
    &= -i \hbar \; \partial_x \psi(x) \; \mbox{ (par propriété de la transformée de Fourier)}
\end{align}

Ceci est bien cohérent avec l'équivalence faite au début du cours entre l'opérateur impulsion $\hat{P}$ et $-i\hbar \; \partial_x$ afin de déterminer l'équation de Schrödinger. \\
Nous pouvons en déduire facilement $\bra{\varphi} \hat{P} \ket{\psi}$ : 

\begin{align}
    \bra{\varphi} \hat{P} \ket{\psi} &= \int dx \; \braket{\varphi | x} \bra{x} \hat{P} \ket{\psi} \notag \\
    &= \int dx \; \varphi^*(x) \left( -i\hbar \; \partial_x \psi(x) \right) \notag \\
\end{align}

Montrons à présent la relation $[\hat{X}, \hat{P}] = i\hbar \; \mathbb{I}$ : 
\begin{align}
    \bra{x}[\hat{X}, \hat{P}] \ket{\psi} &= \bra{x} \hat{X}\hat{P}\ket{\psi} - \bra{x} \hat{P}\hat{X}\ket{\psi} \notag \\
    &= x \bra{x}\hat{P}\ket{\psi} - \bra{x}\hat{P}(\hat{X}\ket{\psi}) = x \bra{x}\hat{P}\ket{\psi} - ( -i\hbar \partial_x \bra{x}\hat{X}\ket{\psi} ) = \notag \\
    &= \left( -i\hbar \partial_x \braket{x|\psi} \right) x + i\hbar \partial_x (x \psi(x)) \notag \\
    &= -i\hbar x \partial_x \psi(x) + i\hbar x\partial_x \psi(x) + i\hbar \psi(x) \notag \\
    &= i\hbar \psi(x) = \bra{x} i\hbar \; \mathbb{I}\ket{\psi} 
\end{align}
Ceci étant valabe pour tout $\ket{x}$, $\ket{\psi}$, nous avons bien que \begin{equation}
     [\hat{X}, \hat{P}] = i\hbar \; \mathbb{I} \end{equation} 

\subsection*{Comment justifier ces notations ?}

Considérons $\mathcal{H} = L^2(\mathbb{R})$ l'espace de Hilbert des fonctions de carré sommables (à une dimension). \\ 
Nous avions mentionné au début du chapitre que pour représenter les opérateurs $\hat{X}$ et $\hat{P}$, nous utiliserons une base dont les éléments n'appartiennent pas à l'espace de Hilbert, et nous avions fait comme s'ils étaient état propre des opérateurs positions $\hat{X}$ et impulsion $\hat{P}$. \\
Nous allons maintenant voir que, ce qu'on appelle le \textit{"Rigged Hilbert Space"} permet de justifier cette démarche ; en effet, cet espace est construit pour relier les notions de distributions et de fonctions de carré sommables en analyse de fonctions. \\

Soit $\ket{\psi}$, $\ket{\varphi}$ $\in \mathcal{H}$ et $A : \mathcal{H} \rightarrow \mathcal{H}$ un opérateur linéaire. \\
Nous savons déjà comment le produit scalaire $\braket{\varphi | \psi}$ peut être défini sur l'espace $\mathcal{H}$ ; de plus, l'opérateur linéaire $A$ permet de calculer des quantités comme $\bra{\varphi} A \ket{\psi}$. \\

Considérons à présent : \begin{description}
    \item[$S =$] $\left\{ \mbox{ fonctions $\mathcal{C}^{\infty}$ à décroissance rapide } \right\} \subset \mathcal{H}$  
    \item[$S^*=$] $\left\{ \mbox{ formes linéaires continues sur $S$ } \right\}$, correspondent aux distributions tempérées, comme mentionné dans le chapitre sur les notions mathématiques. C'est en particulier le dual de $S$. 
\end{description}

Nous pouvons remarquer que $S \subset \mathcal{H} \subset{S^*}$. \\
En effet, il existe un théorème mathématique (appelé \textit{théorème d'Erdös-Kaplansky}, ce sujet est par ailleurs hors du cadre du cours) qui permet de justifier que lorsque la dimension d'un espace vectoriel est infinie, alors aucune application linéaire allant de cet espace à son dual n'est surjective. Autrement dit, nous avons dans notre cas que $\mathcal{H} \subset \mathcal{H}^*$, où l'inclusion est stricte. \\
De plus, $S \subset \mathcal{H} \implies \mathcal{H}^* \subset S^*$ car : il est clair que si nous avons une forme linéaire $f$ agissant sur $\mathcal{H}$ (\textit{i.e} $f \in \mathcal{H}^*$), alors puisque $S \subset \mathcal{H}$, $f$ peut agir en particulier sur $S$, et donc $f \in S^*$. Ainsi, nous avons bien que $\mathcal{H}^* \subset S^*$. En d'autres mots, nous pouvons comprendre cela par le fait que prendre une classe de fonction plus grande (petit), "réduit (augmente)" la taille de son dual. \\

Nous trouvons donc les inclusions suivantes : $S \subset \mathcal{H} \subset \mathcal{H}^* \subset S^*$, et ainsi $S \subset \mathcal{H} \subset S^*$. \\

Prenons à présent $\ket{\phi} \in S$ et $\ket{T} \in S^*$ ; $\braket{T|\phi}$ est bien défini par définition de $S$ et $S^*$. \\
De plus, si $A : S \rightarrow S$ est un opérateur linéaire, alors $A \ket{T}$ est défini par $\bra{\phi}(A\ket{T}) = (\bra{\phi}A)\ket{T}$, pour tout $\ket{T} \in S^*$, $\ket{\phi} \in S$. \\
Ainsi, puisque le produit scalaire hermitien est défini entre $\ket{\phi}$ et $\ket{T}$ appartenant à $S$ et $S^*$ respectivement, il est dès lors possible de représenter un état dans une base de vecteurs n'appartenant pas à $\mathcal{H}$ mais à $S^*$. Nous pouvons donc élargir l'espace de Hilbert à $S^*$. \\

\textit{Remarque :} il reste encore à voir dans quelle mesure nous pouvons définir les grandeurs comme $\braket{T|T'}$ pour $\ket{T}$, $\ket{T'} \in S^*$. 


\subsection*{Opérateur translation}

Soit $\hat{X}$ et $\hat{P}$ les observables positions et impulsions reliées par la relation $[\hat{X}, \hat{P}] = i\hbar \; \mathbb{I}$. Quelque soit $\lambda \in \mathbb{R}$, on définit l'\emph{opérateur translation} $S(\lambda)$ par
$$ S(\lambda) = e^{-i\frac{\lambda \hat{P}}{\hbar}} $$

Cet opérateur vérifie les propriétés suivantes : 
\begin{itemize}[label = \textbullet]
    \item $S(\lambda) S(\lambda') = S(\lambda + \lambda')$  \; $\forall \lambda, \lambda' \in \mathbb{R}$ ; 
    \item $S(\lambda)$ est unitaire : $S^\dagger (\lambda) = S(-\lambda)$ et $S(-\lambda)S(\lambda) = \mathbb{I} \implies S(-\lambda) = S^\dagger(\lambda) = S^{-1}(\lambda)$
\end{itemize}

\textit{Remarque :} Nous avons obtenu les propriétés précédentes en utilisant le fait que pour $A$, $B$ deux opérateurs, on a que $e^A e^B = e^{A+B}$. Ceci est vrai dans les cas où $[A, B] = 0$ (ce qui était vérifié dans notre cas), mais donc n'est pas vrai en général !! Nous allons voir tout de suite une propriété sur les exponentielles de matrices. \\

Nous voulons maintenant déterminer la valeur de $[\hat{X},S(\lambda)]$. \\

Commençons d'abord par voir quelques propriétés sur les commutateurs : 

\begin{enumerate}
    \item Le commutateur est un opérateur bilinéaire, antisymétrique et vérifiant, pour tout opérateur $A$, $B$ et $C$, l'identité de Jacobi donnée par l'expression suivante :
	\begin{equation}
		[[\hat{A},\hat{B}],\hat{C}] + [[\hat{B},\hat{C}],\hat{A}] + [[\hat{C},\hat{A}],\hat{B}] = 0
	\end{equation}
    \item Pour tout opérateur $A$, $B$ et $C$ : 
    \begin{equation}
        [A,BC] = [A,B]C + B[A,C]
        \label{eq:commutateur de produit}
    \end{equation}
    \item $[\hat{X}, \hat{P}^n] = i\hbar n\hat{P}^{n-1}$ ; \\
    Nous pouvons le montrer par récurrence. Commençons par $n=1$ pour l'initialisation : $[\hat{X}, \hat{P}] = i\hbar \; \mathbb{I}$ est vrai par leur définition. Pour montrer l'hérédité, supposons que la propriété est vraie jusque $n$. Nous avons alors que :
    \begin{align} [\hat{X}, \hat{P}^{n+1}] = [\hat{x}, \hat{P}\hat{P}^n] &= [\hat{X}, \hat{P}]\hat{P}^n + \hat{P}[\hat{X}, \hat{P}^n] \; \mbox{ par la propriété \eqref{eq:commutateur de produit}} \notag \\
        &= i\hbar \hat{P}^n + i\hbar n \hat{P}^n \; \mbox{ puisque la propriété est vraie jusque $n$} \notag \\
        &= i\hbar \hat{P}^n (n+1)
    \end{align}
    \item Pour tout opérateurs $A$, $B$ commutant avec $[A,B]$, 
    	\begin{equation}
            e^A e^B = e^{A+B}e^{\frac{1}{2} [A,B]}
    	\end{equation}
    Il s'agit d'un cas particulier de l'identité de Backer-Hausdorff (elle est parfois également appelée formule de Glaubert).
    \item Soit $F(\hat{P}) = \sum_{n=0}^\infty a_n \hat{P}^n$, une fonction de l'observable impulsion. Alors, $[\hat{X}, F(\hat{P})] = i\hbar F'(\hat{P})$. \\
    En effet, \begin{align}
        [\hat{X}, F(\hat{P})] &= [\hat{X}, \sum_n a_n \hat{P}^n] = \sum_n a_n[\hat{X}, \hat{P}^n] \notag \\
        &= \sum_n a_n i\hbar n \hat{P}^{n-1} \notag \\
        &= i\hbar F'(\hat{P}) \label{commutateur d'une fonction de P}
    \end{align}
\end{enumerate}

\'Etant donné que l'opérateur translation $S(\lambda)$ est une fonction de $\hat{P}$, nous pouvons déduire par la dernière propriété énoncée \eqref{commutateur d'une fonction de P} ce que nous voulions calculer au départ : 
$[\hat{X}, S(\lambda)] = i\hbar \left( \frac{-i \lambda}{\hbar} \right) S(\lambda) = \lambda S(\lambda)$. \\
Donc \begin{equation}
    [\hat{X}, S(\lambda)] =\lambda S(\lambda)
    \label{commutateur de S}
\end{equation}

Nous allons essayer de voir en quoi $S(\lambda)$ est un opérateur de \textbf{translation}. \\
Soit $\ket{x_0}$ un vecteur propre de $\hat{X}$ de valeur propre $x_0$ : $$ \hat{X} \ket{x_0} = x_0 \ket{x_0} $$
Calculons $\hat{X} (S(\lambda) \ket{x_0})$ : 
\begin{align}
    \hat{X} (S(\lambda) \ket{x_0}) &= \left( \hat{X}S(\lambda) + S(\lambda) \hat{X} - S(\lambda)\hat{X} \right) \ket{x_0} \notag \\
    &= \left( (\hat{X}S(\lambda) - S(\lambda)\hat{X}) + S(\lambda)\hat{X} \right) \ket{x_0} \notag \\
    &= \left( \lambda S(\lambda) + S(\lambda) \hat{X} \right) \ket{x_0} \; \mbox{ par \eqref{commutateur de S}} \notag \\
    &= (x_0 + \lambda) \; S(\lambda) \ket{x_0}
\end{align}
Nous pouvons donc voir que $(x_0 + \lambda)$ est une valeur propre de l'opérateur position pour le vecteur propre $S(\lambda) \ket{x_0}$. Or puisqu'un vecteur appartenant à un sous-espace propre de $\hat{X}$ de valeur propre $x$ peut s'écrire comme $\ket{x}$, nous pouvons réécrire $S(\lambda) \ket{x_0}$ comme le vecteur $\ket{x_0 + \lambda}$. \\

Cela a pour conséquence que pour $\ket{\psi}$ un ket de fonction d'onde $\braket{x|\psi} = \psi(x)$, alors $S(\lambda) \ket{\psi}$ est un ket de fonction d'onde : 
\begin{align}
    \bra{x}S(\lambda) \ket{\psi} &= (\bra{x}S^\dagger (-\lambda)) \ket{\psi} \notag \\
    &= \braket{x-\lambda | \psi} \notag \\
    &= \psi(x-\lambda) 
\end{align}

Autrement dit, nous pouvons voir que l'opérateur $S(\lambda)$ agit sur les vecteurs de la base position en les translatant d'une valeur de $\lambda$ (et agit de manière inverse sur les composantes d'un vecteur quelconque $\ket{\psi}$, \textit{i.e} $S(\lambda)\ket{\psi}$ est un vecteur de fonction d'onde $\psi(x-\lambda)$). \\
Il est donc cohérent d'appeler $S(\lambda) = \exp \left( -i \frac{\lambda \hat{P}}{\hbar} \right)$ l'opérateur de translation. 

% \section{Valeurs propres et vecteurs propres de Q}
% \subsection{Spectre de Q}

% \begin{Property}
% 	Soit $\ket{x_0}$ le vecteur propre de X, de valeur propre $x_0$. Alors,
% 	\begin{equation}
% 		S(\lambda)\ket{x_0} = \ket{x_0+\lambda}
% 	\end{equation}
% \end{Property}
% \begin{proof}
% 	\begin{align*}
% 		XS(\lambda)\ket{x_0} &= (S(\lambda)X+\lambda S(\lambda))\ket{x_0}\\
% 		&= S(\lambda)x_0\ket{x_0} + \lambda S(\lambda)\ket{x_0} = (x_0+\lambda)S(\lambda)\ket{x_0}
% 	\end{align*}
% \end{proof}

% Cette propriété exprime que $S(\lambda)\ket{x_0}$ est un autre vecteur propre non nul de X, de valeur propre ($x_0+\lambda$). A partir d'un vecteur propre de X, nous pouvons alors en construire un autre : le spectre de X est continu, composé de toutes les valeurs de l'axe réelle.

% \begin{Property}
% 	Si $\ket{\psi}$ est un vecteur de la fonction d'onde $\psi$, alors $S(\lambda)\ket{\psi}$ est un ket d ela fonction d'onde $\psi(x-\lambda)$.
% \end{Property}

% \begin{remark}
% 	Nous avons vu que $S(\lambda)\ket{x_0} = \ket{x_0+\lambda}$. Remarquons que l'expression adjointe s'écrit
% 	\begin{equation}
% 		\bra{x_0}S^\dagger (\lambda) = \bra{x_0+\lambda}
% 	\end{equation}
% 	Soit alors,
% 	\begin{equation}
% 		\bra{x_0}S(\lambda) = \bra{x_0-\lambda}
% 	\end{equation}
% \end{remark}

% \begin{Property}
% 	On remarque alors que si $\ket{\psi}$ est un ket de la fonction d'onde $\psi(x)$, alors $S(\lambda)\ket{\psi}$ est le ket associé à la fonction d'onde $\psi(x-\lambda)$.
% \end{Property}
% \begin{proof}
% 	\begin{align*}
% 		\braket{x|\psi} &= \psi(x)\\
% 		\braket{x|S(\lambda)|\psi} &= \braket{x-\lambda|\psi} = \psi(x-\lambda)
% 	\end{align*}
% \end{proof}

% Ces propriétés de $S(\lambda)$ lui valent le nom de \emph{opérateur de translation}.

% \subsection{Invariance par translation}
% Supposons que le système est invariant par translation, c'est à dire que, pour tout $t,\lambda,\ket{\psi}$:
% \begin{equation}
% 	e^{-i\frac{Ht}{\h}}S(\lambda)\ket{\psi} = S(\lambda)e^{-i\frac{Ht}{\h}}\ket{\psi}
% \end{equation}

% Nous pouvons alors montrer que $HP\ket{\psi} = PH\ket{\psi}$, c'est à dire que $[H,P] = 0$.\\

% L'invariance de translation implique la conservation du générateur des translations ; c'est à dire la conservation de l'impulsion. Ce résultat exploite le Théorème d'Emmy Nöther (\emph{pour une démonstration, le lecteur est invité à suivre le cours de}  $\href{https://www.ulb.be/fr/programme/math-f204}{\text{\emph{Mécanique Analytique - MATH-F204}}}$).

% \section{Relations d'incertitudes}
% Soient A,B des observables et $\ket{\psi}$ un état.

% \begin{remark}
% 	Nous notons $<A^n> = \braket{\psi|A^n|\psi}$, $\Delta A^2 = <A^2> - <A>^2$. De plus, on intoduit $A' = A-<A>$ afin de pouvoir noter $\Delta A^2 = <A'^2>$. On note que $[A,B] = [A',B']$.
% \end{remark}

% \begin{theorem}
% 	Soit A,B deux observables. Alors,
% 	\begin{equation}
% 		\Delta A\Delta B \geq \frac{1}{2} \norm{\braket{[A,B]}}
% 	\end{equation}
% \end{theorem}

% \begin{proof}
% 	\color{red} A compléter.
% \end{proof}
% \color{black}
% Pour les opérateurs X et P, nous avons alors que $[X,P]$ valent $i\h$ ; il s'ensuit que
% \begin{equation*}
% 	\Delta X \Delta P \geq \frac{\h}{2},
% \end{equation*}
% ce qui est exactement la relation $\eqref{Heinsenberg}$.
 
 \end{document}
