\documentclass[../notesdecours.tex]{subfiles}

\begin{document}
\part{Représentations de la position et de l'impulsion en $\mathcal{M}$écanique $\mathcal{Q}$uantique}
Dans ce chapitre, nous allons de nouveau considérer une particule ; en particulier, nous voulons pouvoir définir les notions de position et d'impulsion.\\

Nous allons travailler dans les espaces de Hilbert $L_2(\mathbb{R})$ ou $L_2 (\mathbb{R}^3)$. Nous aurons alors des fonctions de carré sommable. Dans ces espaces, \emph{les opérateurs position X et impulsion P n'ont pas de vecteurs propres}. Nous pouvons néamoins faire comme si ils en avaient : nous expliqueront ultérieurement comment nous pouvons justifier cette approche.\\

Dans cette section, nous utiliserons intensément les résultats obtenus en $\ref{Espace mathématique des fonctions d'onde}$.\\

Introduisons les notations:
\begin{center}
\begin{tabular}{c|c}
$\ket{x_0} $ & Etat propre de l'opérateur X de valeur propre $x_0$. Cela correspond à la "fonction d'onde" $\delta(x-x_0)$.\\
$\ket{p_0} $ & Etat propre de l'opérateur P de valeur propre $p_0$. Cela correspond à la "fonction d'onde" $\frac{1}{\sqrt{2\pi\h}}e^{i \frac{p_0 x}{\h}}$
\end{tabular}
\end{center}

Nous pouvons effectuer plusieurs opérations sur ces objets. 

\paragraph{Normalisation}
Nous voulons calculer $\braket{x_0|x_0'}$ et $\braket{p_0|p_0'}$. 
	\begin{align*}
\braket{x_0|x_0'} &= \int dx \delta (x-x_0)\delta(x-x_0') = \delta (x_0 - x_0)\\
\braket{p_0|p_0'} &= \int dx \frac{e^{-i \frac{p_0 x}{\h}}}{\sqrt{2\pi\h}} \frac{e^{i \frac{p_0'x}{\h}}}{\sqrt{2\pi\h}} = \int du \frac{e^{-i (p_0 - p_0')u}}{2\pi}	= \delta (p_0-p_0')	&u = \frac{x}{\h}\\
	\end{align*}
Ce faisant, nous montrons que les deux bases définies par ces opérations sont orthonormées.

\paragraph{Relation de complétude} \label{complétude}
A partir de là, nous obtenons les relations fondamentales suivantes:
\begin{center}
\begin{tabular}{c|c}
$\braket{x_0|x_0'} = \delta (x_0 - x_0)$ & $\braket{p_0|p_0'} = \delta (p_0-p_0')$\\
$\int d^3 \; \ket{x_0}\bra{x_0} = \mathbb{I}$ & $\int dp_0 \; \ket{p_0}\bra{p_0} = \mathbb{I}$
\end{tabular}
\end{center}
Nous avons alors deux relations de \emph{complétude}, ou de \emph{fermeture}.

\paragraph{Composante d'un ket}
Considérons un état quantique $\ket{\Psi}$, correspondant à la fonction d'onde $\Psi(x)$. En exploitant les relations de fermetures définies ci-dessus, nous pouvons alors écrire l'état quantique sous les deux formes suivantes:
\begin{align}
\ket{\Psi} &= \int d^3x_0 \; \ket{x_0} \bra{x_0} \ket{\Psi}\\
\ket{\Psi} &= \int d^3p_0 \; \ket{p_0} \bra{p_0} \ket{\Psi}
\end{align}
On pose $\psi(x) = \braket{x|\psi}$. Observons que
\begin{align}
\ket{\psi} &= \int dx \; \ket{x}\bra{x} \; \ket{\psi} = \int dx \; \ket{X} \braket{x|\psi}\\
&= \int \psi(x)\ket{x}.
\end{align}
En particulier, en prenant $\ket{\psi} = \ket{p}$, nous avons que
\begin{equation}
\braket{x|p} = \frac{1}{\sqrt{2\pi\h}}e^{i \frac{px}{\h}}
\end{equation}
Dès lors, 
\begin{align}
\braket{p|\psi} &= \braket{p|\mathbb{I}|\psi}\notag\\
&= \braket{p|\int dx \ket{x}\bra{x} \ket{\psi}}\notag\\
&= \int dx \braket{p|x}\braket{x|\psi}\notag\\
\braket{p|\psi} &= \frac{1}{\sqrt{2\pi\h}}\int dx \; e^{i \frac{px}{\h}}\psi(x) = \tilde{\psi}(p)
\end{align}
Où $\tilde{\psi}(p)$ est par définition la transformée de Fourier de $\psi(r)$.\\

Pour résumer, nous avons que 
\begin{align}
\braket{\bm{r}|\psi} &= \psi(\bm{r})\\
\braket{\bm{p}|\psi} &= \tilde{\psi}(\bm{p})
\end{align}
\paragraph{Produit scalaire de deux vecteurs}
En vertue des relations de complétude $\ref{complétude}$, il est possible de retrouver le produit scalaire $\eqref{Produit scalaire - Espace des fonctions d'onde}$.

\begin{align}
\braket{\varphi|\psi} &= \braket{\varphi|\int dx \; \ket{x}\bra{x}|\ket{\psi}} & \braket{\varphi|\psi} &= \braket{\varphi|\int dp \; \ket{p}\bra{p}|\psi}\notag\\
&= \int dx \; \braket{\varphi|x}\braket{x|\psi} & &= \int dp \; \braket{\varphi|p}\braket{p|\psi}\notag\\
\braket{\varphi|\psi} &= \int dx \; \varphi^*(x) \psi (x) & \braket{\varphi|\psi} &= \int dp \tilde{\varphi}^* (p) \tilde{\psi}(p) 
\end{align}

\paragraph{Opérateurs X et P}
Soit $\ket{\psi}$ un ket quelconque et $\braket{\bm{r}|\psi} \doteq \psi (x,y,z)$ la fonction d'onde correspondante. On définit l'opérateur X de sorte que
\begin{equation}
\ket{\psi'} = X\ket{\psi}
\end{equation}
soit définit à travers la base $\{\bm{r}\}$ par la fonction $\braket{\bm{r}|\psi'} = \psi'(\bm{r})$, où
\begin{equation}
\psi' (\bm{r}) = x\psi(x,y,z).
\end{equation}
Dans cette base, l'opérateur X représente donc la multiplication par x. De manière analogue, nous introduisons les opérateurs Y et Z:
\begin{align}
\braket{\bm{r}|X|\psi} &= x\braket{\bm{r}|\psi} & \braket{\bm{r}|Y|\psi} &= y\braket{\bm{r}|\psi} & \braket{\bm{r}|Z|\psi} &= z\braket{\bm{r}|\psi}
\end{align}
Similairement, on définit l'opérateur $\bm{P}$, dont l'action dans la base $\ket{\bm{p}}$ est donnée par
\begin{align}
\braket{\bm{p}|P_x|\psi} &= p_x \braket{\bm{p}|\psi} = p_x\tilde{\psi}(p_x) & \braket{\bm{p}|P_y|\psi} &= p_y \braket{\bm{p}|\psi} = p_y\tilde{\psi} (p_y) & \braket{\bm{p}|P_z|\psi} &= p_z \braket{\bm{p}|\psi} = p_z\tilde{\psi}(p_z)
\end{align}
\begin{Property} $\braket{x|P|\psi} = -i\h\partial_x\braket{x|\psi}$. \end{Property}
\begin{proof}
\begin{align}
\braket{x|P|\psi} &= \int dp \; \braket{x|p}\braket{p|P|\psi}\notag\\
&= \int dp \; \frac{e^{i \frac{px}{\h}}}{\sqrt{2\pi\h}}p\tilde{\psi} (p)\notag\\
&= -i\h \partial_x (\int dp \; \frac{e^{i \frac{px}{\h}}}{\sqrt{2\pi\h}} \tilde{\psi} (p)\notag\\
\braket{x|P|\psi} &=-i\h\partial_x\psi(x)
\end{align}
\end{proof}

\begin{Property} $[X,P] = i\h\mathbb{I}$ \end{Property}
\begin{proof}
La preuve est détaillée dans le livre de référence.
\end{proof}
Nous pouvons en déduire que 
\begin{align}
[R_i,R_j] &= 0 & [P_i,P_j] &= 0 & [R_i,P_j] = i\h\delta_{ij}
\end{align}
\end{document}