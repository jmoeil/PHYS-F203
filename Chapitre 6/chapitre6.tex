\documentclass[../Notesdecours.tex]{subfiles}

\begin{document}
\part{Représentations de la position et de l'impulsion en $\mathcal{M}$écanique $\mathcal{Q}$uantique}
\label[]{Chapitre 6}
Dans ce chapitre, nous allons de nouveau considérer une particule ; en particulier, nous voulons pouvoir définir les notions de position et d'impulsion.\\

Nous allons travailler dans les espaces de Hilbert $L_2(\mathbb{R})$ ou $L_2 (\mathbb{R}^3)$. Nous aurons alors des fonctions de carré sommable. Dans ces espaces, \emph{les opérateurs position X et impulsion P n'ont pas de vecteurs propres}. Nous pouvons néamoins faire comme si ils en avaient : nous expliqueront ultérieurement comment nous pouvons justifier cette approche.\\

Dans cette section, nous utiliserons intensément les résultats obtenus en $\ref{Espace mathématique des fonctions d'onde}$.\\

Introduisons les notations:
\begin{center}
\begin{tabular}{c|c}
$\ket{x_0} $ & Etat propre de l'opérateur X de valeur propre $x_0$. Cela correspond à la "fonction d'onde" $\delta(x-x_0)$.\\
$\ket{p_0} $ & Etat propre de l'opérateur P de valeur propre $p_0$. Cela correspond à la "fonction d'onde" $\frac{1}{\sqrt{2\pi\h}}e^{i \frac{p_0 x}{\h}}$
\end{tabular}
\end{center}

Nous pouvons effectuer plusieurs opérations sur ces objets. 

\section{Espace vectoriel des opérateurs X et P}
\subsection{Normalisation}
Nous voulons calculer $\braket{x_0|x_0'}$ et $\braket{p_0|p_0'}$. 
	\begin{align*}
\braket{x_0|x_0'} &= \int dx \delta (x-x_0)\delta(x-x_0') = \delta (x_0 - x_0)\\
\braket{p_0|p_0'} &= \int dx \frac{e^{-i \frac{p_0 x}{\h}}}{\sqrt{2\pi\h}} \frac{e^{i \frac{p_0'x}{\h}}}{\sqrt{2\pi\h}} = \int du \frac{e^{-i (p_0 - p_0')u}}{2\pi}	= \delta (p_0-p_0')	&u = \frac{x}{\h}\\
	\end{align*}
Ce faisant, nous montrons que les deux bases définies par ces opérations sont orthonormées.

\subsection{Relation de complétude} \label{complétude}
A partir de là, nous obtenons les relations fondamentales suivantes:
\begin{center}
\begin{tabular}{c|c}
$\braket{x_0|x_0'} = \delta (x_0 - x_0)$ & $\braket{p_0|p_0'} = \delta (p_0-p_0')$\\
$\int d^3 \; \ket{x_0}\bra{x_0} = \mathbb{I}$ & $\int dp_0 \; \ket{p_0}\bra{p_0} = \mathbb{I}$
\end{tabular}
\end{center}
Nous avons alors deux relations de \emph{complétude}, ou de \emph{fermeture}.

\subsection{Composante d'un ket}
Considérons un état quantique $\ket{\Psi}$, correspondant à la fonction d'onde $\Psi(x)$. En exploitant les relations de fermetures définies ci-dessus, nous pouvons alors écrire l'état quantique sous les deux formes suivantes:
\begin{align}
\ket{\Psi} &= \int d^3x_0 \; \ket{x_0} \bra{x_0} \ket{\Psi}\\
\ket{\Psi} &= \int d^3p_0 \; \ket{p_0} \bra{p_0} \ket{\Psi}
\end{align}
On pose $\psi(x) = \braket{x|\psi}$. Observons que
\begin{align}
\ket{\psi} &= \int dx \; \ket{x}\bra{x} \; \ket{\psi} = \int dx \; \ket{X} \braket{x|\psi}\\
&= \int \psi(x)\ket{x}.
\end{align}
En particulier, en prenant $\ket{\psi} = \ket{p}$, nous avons que
\begin{equation}
\braket{x|p} = \frac{1}{\sqrt{2\pi\h}}e^{i \frac{px}{\h}}
\end{equation}
Dès lors, 
\begin{align}
\braket{p|\psi} &= \braket{p|\mathbb{I}|\psi}\notag\\
&= \braket{p|\int dx \ket{x}\bra{x} \ket{\psi}}\notag\\
&= \int dx \braket{p|x}\braket{x|\psi}\notag\\
\braket{p|\psi} &= \frac{1}{\sqrt{2\pi\h}}\int dx \; e^{i \frac{px}{\h}}\psi(x) = \tilde{\psi}(p)
\end{align}
Où $\tilde{\psi}(p)$ est par définition la transformée de Fourier de $\psi(r)$.\\

Pour résumer, nous avons que 
\begin{align}
\braket{\bm{r}|\psi} &= \psi(\bm{r})\\
\braket{\bm{p}|\psi} &= \tilde{\psi}(\bm{p})
\end{align}
\subsection{Produit scalaire de deux vecteurs}
En vertue des relations de complétude $\ref{complétude}$, il est possible de retrouver le produit scalaire $\eqref{Produit scalaire - Espace des fonctions d'onde}$.

\begin{align}
\braket{\varphi|\psi} &= \braket{\varphi|\int dx \; \ket{x}\bra{x}|\ket{\psi}} & \braket{\varphi|\psi} &= \braket{\varphi|\int dp \; \ket{p}\bra{p}|\psi}\notag\\
&= \int dx \; \braket{\varphi|x}\braket{x|\psi} & &= \int dp \; \braket{\varphi|p}\braket{p|\psi}\notag\\
\braket{\varphi|\psi} &= \int dx \; \varphi^*(x) \psi (x) & \braket{\varphi|\psi} &= \int dp \tilde{\varphi}^* (p) \tilde{\psi}(p) 
\end{align}

\section{Opérateurs X et P}
Soit $\ket{\psi}$ un ket quelconque et $\braket{\bm{r}|\psi} \doteq \psi (x,y,z)$ la fonction d'onde correspondante. On définit l'opérateur X de sorte que
\begin{equation}
\ket{\psi'} = X\ket{\psi}
\end{equation}
soit définit à travers la base $\{\bm{r}\}$ par la fonction $\braket{\bm{r}|\psi'} = \psi'(\bm{r})$, où
\begin{equation}
\psi' (\bm{r}) = x\psi(x,y,z).
\end{equation}
Dans cette base, l'opérateur X représente donc la multiplication par x. De manière analogue, nous introduisons les opérateurs Y et Z:
\begin{align}
\braket{\bm{r}|X|\psi} &= x\braket{\bm{r}|\psi} & \braket{\bm{r}|Y|\psi} &= y\braket{\bm{r}|\psi} & \braket{\bm{r}|Z|\psi} &= z\braket{\bm{r}|\psi}
\end{align}
Similairement, on définit l'opérateur $\bm{P}$, dont l'action dans la base $\ket{\bm{p}}$ est donnée par
\begin{align}
\braket{\bm{p}|P_x|\psi} &= p_x \braket{\bm{p}|\psi} = p_x\tilde{\psi}(p_x) & \braket{\bm{p}|P_y|\psi} &= p_y \braket{\bm{p}|\psi} = p_y\tilde{\psi} (p_y) & \braket{\bm{p}|P_z|\psi} &= p_z \braket{\bm{p}|\psi} = p_z\tilde{\psi}(p_z)
\end{align}
\begin{Property} $\braket{x|P|\psi} = -i\h\partial_x\braket{x|\psi}$. \end{Property}
\begin{proof}
\begin{align}
\braket{x|P|\psi} &= \int dp \; \braket{x|p}\braket{p|P|\psi}\notag\\
&= \int dp \; \frac{e^{i \frac{px}{\h}}}{\sqrt{2\pi\h}}p\tilde{\psi} (p)\notag\\
&= -i\h \partial_x (\int dp \; \frac{e^{i \frac{px}{\h}}}{\sqrt{2\pi\h}} \tilde{\psi} (p)\notag\\
\braket{x|P|\psi} &=-i\h\partial_x\psi(x)
\end{align}
\end{proof}

\begin{Property} $[X,P] = i\h\mathbb{I}$ \end{Property}
\begin{proof}
La preuve est assez simple:
\begin{align*}
	\braket{\bm{r}|[X,P]|\psi} &= \braket{\bm{r}|XP-PX|\psi}\\
	&= \braket{\bm{r}|XP|\psi} - \braket{\bm{r}|PX|\psi}\\
	&= x\braket{\bm{r}|P|\psi} -\frac{\h}{i}\frac{\partial}{\partial x}\braket{\bm{r}|X|\psi}\\
	&= \frac{\h x}{i}\frac{\partial}{\partial x}\braket{\bm{r}|\psi} - \frac{\h}{i} \frac{\partial}{\partial x}x\braket{\bm{r}|\psi}\\
	&= i\h \braket{\bm{r}|\psi}
\end{align*}
Cela étant vrai pour tout $\bm{r}$ et tout $\psi$, il s'ensuit que $[X,P] = i\h$.
\end{proof}
Nous pouvons en déduire que 
\begin{align}
[X_i,X_j] &= 0 & [P_i,P_j] &= 0 & [X_i,P_j] = i\h\delta_{ij}
\end{align}
Pour tout $i,j = 1,2,3$.

\section{Opérateur translation}
\begin{definition}
	Soit $P,Q$, deux observables reliées par la relation $[P,Q] = i\h\mathbb{I}$. On définit l'\emph{opérateur translation} $S(\lambda)$ par
	\begin{equation}
		S(\lambda) = e^{-i\frac{\lambda P}{\h}}
	\end{equation}
	Pour tout $\lambda\in\mathbb{R}$.
\end{definition}
Observons que cet opérateur est unitaire : effectivement, $S^\dagger (\lambda) = S(-\lambda)$. De plus, $S(\lambda)S(\lambda') = S(\lambda + \lambda')$.\\

Nous voulons déterminer la valeur de $[X,S(\lambda)]$. Commecons par remarquer les propriétés suivantes de l'opérateur commutateur.
\subsection{Quelques propriétés de l'opétateur commutateur}
\begin{Property}
	Pour tout opérateur A,B et C,
	\begin{equation}
		[A,BC] = [A,B]C + B[A,C]
	\end{equation}
\end{Property}
\begin{Property}
	\label[]{Commutateur dérivée}
	Pour tout opérateur X et P et tout naturel n,
	\begin{equation}
		[X,P^n] = i\h nP^{n-1}
	\end{equation}
\end{Property}
\begin{proof}
	Il s'agit d'une preuve par réccurence. La base se prouve assez facilement ; passons directement à l'étape d'induction:
	\begin{equation*}
		[X,P^{n+1}] = [X,P^n \; P] = [X,P^n]P + P^n[X,P] = i\h nP^n + i\h P^n = i\h (n+1)P^n
	\end{equation*}
\end{proof}
Généralisons $\ref{Commutateur dérivée}$ à une fonction pouvant être représentée comme une série, supposée convergente.
\begin{Property}
\label[]{Lien commutateur dérivée}
	Soit $F(P) = \sum_{n = 0}^\infty a_nP^n$ une fonction de P. Alors,
	\begin{equation}
		[X,F(P)] = i\h F'(P)
	\end{equation}
\end{Property}
\begin{proof}
	\begin{align*}
		[X,F(P)] &= \sum_n a_n [X,P^n] = \sum_n a_n i\h nP^{n-} = i\h F'(P)
	\end{align*}
\end{proof}

Cette dernière proposition permet de répondre à la question posée : 
\begin{equation}
	[X,S(\lambda)] = \lambda S(\lambda).	
\end{equation}
 Nous pouvons reformuler cette égalité sous la forme
\begin{equation}
	QS(\lambda) = S(\lambda)[Q+\lambda].
\end{equation}

\section{Valeurs propres et vecteurs propres de Q}
\subsection{Spectre de Q}

\begin{Property}
	Soit $\ket{x_0}$ le vecteur propre de X, de valeur propre $x_0$. Alors,
	\begin{equation}
		S(\lambda)\ket{x_0} = \ket{x_0+\lambda}
	\end{equation}
\end{Property}
\begin{proof}
	\begin{align*}
		XS(\lambda)\ket{x_0} &= (S(\lambda)X+\lambda S(\lambda))\ket{x_0}\\
		&= S(\lambda)x_0\ket{x_0} + \lambda S(\lambda)\ket{x_0} = (x_0+\lambda)S(\lambda)\ket{x_0}
	\end{align*}
\end{proof}

Cette propriété exprime que $S(\lambda)\ket{x_0}$ est un autre vecteur propre non nul de X, de valeur propre ($x_0+\lambda$). A partir d'un vecteur propre de X, nous pouvons alors en construire un autre : le spectre de X est continu, composé de toutes les valeurs de l'axe réelle.

\begin{Property}
	Si $\ket{\psi}$ est un vecteur de la fonction d'onde $\Psi$, alors $S(\lambda)\ket{\psi}$ est un ket d ela fonction d'onde $\Psi(x-\lambda)$.
\end{Property}

\begin{remark}
	Nous avons vu que $S(\lambda)\ket{x_0} = \ket{x_0+\lambda}$. Remarquons que l'expression adjointe s'écrit
	\begin{equation}
		\bra{x_0}S^\dagger (\lambda) = \bra{x_0+\lambda}
	\end{equation}
	Soit alors,
	\begin{equation}
		\bra{x_0}S(\lambda) = \bra{x_0-\lambda}
	\end{equation}
\end{remark}

\begin{Property}
	On remarque alors que si $\ket{\psi}$ est un ket de la fonction d'onde $\Psi(x)$, alors $S(\lambda)\ket{\psi}$ est le ket associé à la fonction d'onde $\Psi(x-\lambda)$.
\end{Property}
\begin{proof}
	\begin{align*}
		\braket{x|\psi} &= \Psi(x)\\
		\braket{x|S(\lambda)|\psi} &= \braket{x-\lambda|\psi} = \Psi(x-\lambda)
	\end{align*}
\end{proof}

Ces propriétés de $S(\lambda)$ lui valent le nom de \emph{opérateur de translation}.

\subsection{Invariance par translation}
Supposons que le système est invariant par translation, c'est à dire que, pour tout $t,\lambda,\ket{\psi}$:
\begin{equation}
	e^{-i\frac{Ht}{\h}}S(\lambda)\ket{\psi} = S(\lambda)e^{-i\frac{Ht}{\h}}\ket{\psi}
\end{equation}

Nous pouvons alors montrer que $HP\ket{\psi} = PH\ket{\psi}$, c'est à dire que $[H,P] = 0$.\\

L'invariance de translation implique la conservation du générateur des translations ; c'est à dire la conservation de l'impulsion. Ce résultat (\emph{non démontré dans le cadre de ce cours}) exploite le Théorème d'Emmy Nöther.

\section{Relations d'incertitudes}
Soient A,B des observables et $\ket{\psi}$ un état.

\begin{remark}
	Nous notons $<A^n> = \braket{\psi|A^n|\psi}$, $\Delta A^2 = <A^2> - <A>^2$. De plus, on intoduit $A' = A-<A>$ afin de pouvoir noter $\Delta A^2 = <A'^2>$. On note que $[A,B] = [A',B']$.
\end{remark}

\begin{theorem}
	Soit A,B deux obserables. Alors,
	\begin{equation}
		\Delta A\Delta B \geq \frac{1}{2} \norm{<[A,B]>}
	\end{equation}
\end{theorem}

\begin{proof}
	La preuve est laissée en exercice pour le lecteur.
\end{proof}

Pour les opérateurs X et P, nous avons alors que $[X,P]$ valent $i\h$ ; il s'ensuit que
\begin{equation*}
	\Delta X \Delta P \geq \frac{\h}{2},
\end{equation*}
ce qui est exactement la relation $\eqref{Heinsenberg}$.
\end{document}